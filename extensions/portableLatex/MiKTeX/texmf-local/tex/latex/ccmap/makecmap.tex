% $Id: makecmap.tex,v 1.3 2005/11/21 14:57:05 zlb Exp $
%
%  Filename:  makecmap.tex (provided by SUN Wenchang)
%
%  Function:  make cmap files for using cmap.sty with CJK fonts.
%
%  Usage:
%  for GB encoding
%
%           Latex  \def\cmapEnc{GB} % $Id: makecmap.tex,v 1.3 2005/11/21 14:57:05 zlb Exp $
%
%  Filename:  makecmap.tex (provided by SUN Wenchang)
%
%  Function:  make cmap files for using cmap.sty with CJK fonts.
%
%  Usage:
%  for GB encoding
%
%           Latex  \def\cmapEnc{GB} % $Id: makecmap.tex,v 1.3 2005/11/21 14:57:05 zlb Exp $
%
%  Filename:  makecmap.tex (provided by SUN Wenchang)
%
%  Function:  make cmap files for using cmap.sty with CJK fonts.
%
%  Usage:
%  for GB encoding
%
%           Latex  \def\cmapEnc{GB} % $Id: makecmap.tex,v 1.3 2005/11/21 14:57:05 zlb Exp $
%
%  Filename:  makecmap.tex (provided by SUN Wenchang)
%
%  Function:  make cmap files for using cmap.sty with CJK fonts.
%
%  Usage:
%  for GB encoding
%
%           Latex  \def\cmapEnc{GB} \input{makecmap.tex}
%
%       Then you will get 35 or 36 cmap files like c1001.cmap, c1002.cmap, ...
%
%
\documentclass{article}
\usepackage{CJK}
\begin{document}
\makeatletter

\newwrite\cmapout
\newif\if@firstplane
\@firstplanetrue

\newif\if@CJKplane@hexmode
\@CJKplane@hexmodefalse

\newcount\unicode@a
\newcount\unicode@b

\def\unicodecmap{
\@CJKplane@hexmodetrue
\unicode@a 0
\loop
  \CJK@numbToHex{\tempba}{\unicode@a}
  \ifnum\the\unicode@a < 256
     {\output@unicodeplane}
  \advance\unicode@a 1
\repeat
}

\def\output@unicodeplane
{{
\unicode@b 0
\loop
  \ifnum\the\unicode@b < 256
     \CJK@numbToHex{\tempbb}{\unicode@b}
     \edef\tempbb{\tempba\tempbb}
     \expandafter\@@cidmap\tempbb
  \advance\unicode@b 1
\repeat
}
}




\newcount\cmaplines
\newcount\cmapCJKplane
\newcount\cmaptemp
\newcount\cmap@num@a
\newcount\cmap@num@b

\cmapCJKplane 0
\cmaplines 0

\catcode`\%=11
\catcode`\#=14
\catcode`\$=6

\catcode`\x=13
\def x{\@@cidmap}

\def\make@eks@active{
    \catcode`\x=13
    \def x{\@@cidmap}}

\catcode`\x=11

\def\write@cmapheader{
   \begingroup
      \if@filesw
        \immediate\openout \cmapout  \lower@@@enc\cmap@CJKplane.cmap\relax
      \fi
      \@nobreakfalse
      \typeout{...generating \lower@@@enc\cmap@CJKplane.cmap...}
   \endgroup

   \immediate\write\cmapout{%!PS-Adobe-3.0 Resource-CMap}
   \immediate\write\cmapout{%%DocumentNeededResources: ProcSet (CIDInit)}
   \immediate\write\cmapout{%%IncludeResource: ProcSet (CIDInit)}
   \immediate\write\cmapout{%%BeginResource: CMap (TeX-\lower@@@enc-\cmap@CJKplane-0)}
   \immediate\write\cmapout{%%Title: (TeX-\lower@@@enc-\cmap@CJKplane-0\space
       TeX \lower@@@enc-\cmap@CJKplane 0)}
   \immediate\write\cmapout{%%Version: 1.000}
   \immediate\write\cmapout{%%EndComments}
   \immediate\write\cmapout{/CIDInit /ProcSet findresource begin}
   \immediate\write\cmapout{begincmap}
   \immediate\write\cmapout{12 dict begin}
   \immediate\write\cmapout{/CIDSystemInfo}
   \immediate\write\cmapout{<< /Registry (TeX)}
   \immediate\write\cmapout{/Ordering (\lower@@@enc-\cmap@CJKplane)}
   \immediate\write\cmapout{/Supplement 0}
   \immediate\write\cmapout{>> def}
   \immediate\write\cmapout{/CMapName /TeX-\lower@@@enc-\cmap@CJKplane-0 def}
   \immediate\write\cmapout{/CMapType 2 def}
}

\def\write@cmaptotal{
   \cmaptemp\cmaplines
   \advance\cmaptemp -1
   \CJK@numbToHex{\tempba}{\cmaptemp}
   \immediate\write\cmapout{1 begincodespacerange}
   \immediate\write\cmapout{<00> <\tempba>}
   \immediate\write\cmapout{endcodespacerange}
}

\def\write@cmapcontents{

   \cmaptemp\cmap@num@b
   \advance\cmaptemp -\cmap@num@a
   \immediate\write\cmapout{\the\cmaptemp\space beginbfchar}

   {\cmaptemp \cmap@num@a
   \loop
      \ifnum\the\cmaptemp<\cmap@num@b
      \immediate\write\cmapout{\csname cmapline@\the\cmaptemp\endcsname}
      \advance\cmaptemp 1
   \repeat
   \immediate\write\cmapout{endbfchar}
   }
}

\def\write@cmapend{
   \immediate\write\cmapout{endcmap}
   \immediate\write\cmapout{CMapName currentdict /CMap defineresource pop}
   \immediate\write\cmapout{end}
   \immediate\write\cmapout{end}
   \immediate\write\cmapout{%%EndResource}
   \immediate\write\cmapout{%%EOF}
   \immediate\closeout\cmapout
}

\def\@@cidmap$1$2$3$4{

\catcode`\x=11

\ifnum\the\cmaplines=0
   \if@firstplane
      \global\@firstplanefalse
      \ifnum $1=0
      \else
          \global\advance \cmapCJKplane 1
      \fi
   \else
          \global\advance \cmapCJKplane 1
   \fi
   \if@CJKplane@hexmode
      \CJK@numbToHex{\cmap@CJKplane}{\cmapCJKplane}
      \edef\tempa{\lowercase{\def\noexpand\cmap@CJKplane{\cmap@CJKplane}}}
      \tempa
   \else
      \ifnum\the\cmapCJKplane<10
            \edef\cmap@CJKplane{0\the\cmapCJKplane}
      \else
            \edef\cmap@CJKplane{\the\cmapCJKplane}
      \fi
   \fi
\fi
\CJK@numbToHex{\cmap@hexind}{\cmaplines}
\expandafter\edef\csname cmapline@\the\cmaplines\endcsname{<\cmap@hexind> <$1$2$3$4>}
\advance\cmaplines 1

\ifnum\the\cmaplines=256

   \write@cmapheader
   \write@cmaptotal

   \cmap@num@a 0
   \cmap@num@b 100
   \write@cmapcontents

   \cmap@num@a 100
   \cmap@num@b 200
   \write@cmapcontents

   \cmap@num@a 200
   \cmap@num@b 256
   \write@cmapcontents

   \write@cmapend
   \global\cmaplines 0
\fi

\make@eks@active
}


\def\get@sfd$1{
\edef\temp{$1}
\edef\tempa{\lowercase{\def\noexpand\tempa{\temp}}}
\tempa
\edef\sfdname{\csname sfd@\tempa\endcsname}
}

\@namedef{sfd@bg5}{UBig5}
\@namedef{sfd@bg5+}{UBg5plus}
\@namedef{sfd@gb}{UGB}
\@namedef{sfd@gbk}{UGBK}
\@namedef{sfd@jis}{UJIS}
\@namedef{sfd@ks}{UKS}
\@namedef{sfd@utf8}{Unicode}


\def\makecmapfiles$1{
\CJKenc{$1}
\edef\lower@@@enc{\lowercase{\def\noexpand\lower@@@enc{\csname CJK@$1@nfssenc\endcsname}}}
\lower@@@enc
\get@sfd{$1}
\expandafter\ifx\csname \sfdname cmap\endcsname\relax
   \make@eks@active
   \input{\sfdname.sfd}
\else
   \csname \sfdname cmap\endcsname
\fi
}

\makecmapfiles{\cmapEnc}



\catcode`\x=11

\ifnum\the\cmaplines<256
   \ifnum\the\cmap@num@b > 0
      \write@cmapheader
      \write@cmaptotal
   \fi

   \cmap@num@a 0
   \cmap@num@b \cmaplines

   \ifnum\the\cmap@num@b > 100
      \cmap@num@b 100
      \write@cmapcontents
      \cmap@num@a 100
      \cmap@num@b \cmaplines
   \fi

   \ifnum\the\cmap@num@b > 200
      \cmap@num@b 200
      \write@cmapcontents
      \cmap@num@a 200
      \cmap@num@b \cmaplines
   \fi

   \ifnum\the\cmap@num@b > 0
      \write@cmapcontents
      \write@cmapend
   \fi
\fi





\end{document}

%
%       Then you will get 35 or 36 cmap files like c1001.cmap, c1002.cmap, ...
%
%
\documentclass{article}
\usepackage{CJK}
\begin{document}
\makeatletter

\newwrite\cmapout
\newif\if@firstplane
\@firstplanetrue

\newif\if@CJKplane@hexmode
\@CJKplane@hexmodefalse

\newcount\unicode@a
\newcount\unicode@b

\def\unicodecmap{
\@CJKplane@hexmodetrue
\unicode@a 0
\loop
  \CJK@numbToHex{\tempba}{\unicode@a}
  \ifnum\the\unicode@a < 256
     {\output@unicodeplane}
  \advance\unicode@a 1
\repeat
}

\def\output@unicodeplane
{{
\unicode@b 0
\loop
  \ifnum\the\unicode@b < 256
     \CJK@numbToHex{\tempbb}{\unicode@b}
     \edef\tempbb{\tempba\tempbb}
     \expandafter\@@cidmap\tempbb
  \advance\unicode@b 1
\repeat
}
}




\newcount\cmaplines
\newcount\cmapCJKplane
\newcount\cmaptemp
\newcount\cmap@num@a
\newcount\cmap@num@b

\cmapCJKplane 0
\cmaplines 0

\catcode`\%=11
\catcode`\#=14
\catcode`\$=6

\catcode`\x=13
\def x{\@@cidmap}

\def\make@eks@active{
    \catcode`\x=13
    \def x{\@@cidmap}}

\catcode`\x=11

\def\write@cmapheader{
   \begingroup
      \if@filesw
        \immediate\openout \cmapout  \lower@@@enc\cmap@CJKplane.cmap\relax
      \fi
      \@nobreakfalse
      \typeout{...generating \lower@@@enc\cmap@CJKplane.cmap...}
   \endgroup

   \immediate\write\cmapout{%!PS-Adobe-3.0 Resource-CMap}
   \immediate\write\cmapout{%%DocumentNeededResources: ProcSet (CIDInit)}
   \immediate\write\cmapout{%%IncludeResource: ProcSet (CIDInit)}
   \immediate\write\cmapout{%%BeginResource: CMap (TeX-\lower@@@enc-\cmap@CJKplane-0)}
   \immediate\write\cmapout{%%Title: (TeX-\lower@@@enc-\cmap@CJKplane-0\space
       TeX \lower@@@enc-\cmap@CJKplane 0)}
   \immediate\write\cmapout{%%Version: 1.000}
   \immediate\write\cmapout{%%EndComments}
   \immediate\write\cmapout{/CIDInit /ProcSet findresource begin}
   \immediate\write\cmapout{begincmap}
   \immediate\write\cmapout{12 dict begin}
   \immediate\write\cmapout{/CIDSystemInfo}
   \immediate\write\cmapout{<< /Registry (TeX)}
   \immediate\write\cmapout{/Ordering (\lower@@@enc-\cmap@CJKplane)}
   \immediate\write\cmapout{/Supplement 0}
   \immediate\write\cmapout{>> def}
   \immediate\write\cmapout{/CMapName /TeX-\lower@@@enc-\cmap@CJKplane-0 def}
   \immediate\write\cmapout{/CMapType 2 def}
}

\def\write@cmaptotal{
   \cmaptemp\cmaplines
   \advance\cmaptemp -1
   \CJK@numbToHex{\tempba}{\cmaptemp}
   \immediate\write\cmapout{1 begincodespacerange}
   \immediate\write\cmapout{<00> <\tempba>}
   \immediate\write\cmapout{endcodespacerange}
}

\def\write@cmapcontents{

   \cmaptemp\cmap@num@b
   \advance\cmaptemp -\cmap@num@a
   \immediate\write\cmapout{\the\cmaptemp\space beginbfchar}

   {\cmaptemp \cmap@num@a
   \loop
      \ifnum\the\cmaptemp<\cmap@num@b
      \immediate\write\cmapout{\csname cmapline@\the\cmaptemp\endcsname}
      \advance\cmaptemp 1
   \repeat
   \immediate\write\cmapout{endbfchar}
   }
}

\def\write@cmapend{
   \immediate\write\cmapout{endcmap}
   \immediate\write\cmapout{CMapName currentdict /CMap defineresource pop}
   \immediate\write\cmapout{end}
   \immediate\write\cmapout{end}
   \immediate\write\cmapout{%%EndResource}
   \immediate\write\cmapout{%%EOF}
   \immediate\closeout\cmapout
}

\def\@@cidmap$1$2$3$4{

\catcode`\x=11

\ifnum\the\cmaplines=0
   \if@firstplane
      \global\@firstplanefalse
      \ifnum $1=0
      \else
          \global\advance \cmapCJKplane 1
      \fi
   \else
          \global\advance \cmapCJKplane 1
   \fi
   \if@CJKplane@hexmode
      \CJK@numbToHex{\cmap@CJKplane}{\cmapCJKplane}
      \edef\tempa{\lowercase{\def\noexpand\cmap@CJKplane{\cmap@CJKplane}}}
      \tempa
   \else
      \ifnum\the\cmapCJKplane<10
            \edef\cmap@CJKplane{0\the\cmapCJKplane}
      \else
            \edef\cmap@CJKplane{\the\cmapCJKplane}
      \fi
   \fi
\fi
\CJK@numbToHex{\cmap@hexind}{\cmaplines}
\expandafter\edef\csname cmapline@\the\cmaplines\endcsname{<\cmap@hexind> <$1$2$3$4>}
\advance\cmaplines 1

\ifnum\the\cmaplines=256

   \write@cmapheader
   \write@cmaptotal

   \cmap@num@a 0
   \cmap@num@b 100
   \write@cmapcontents

   \cmap@num@a 100
   \cmap@num@b 200
   \write@cmapcontents

   \cmap@num@a 200
   \cmap@num@b 256
   \write@cmapcontents

   \write@cmapend
   \global\cmaplines 0
\fi

\make@eks@active
}


\def\get@sfd$1{
\edef\temp{$1}
\edef\tempa{\lowercase{\def\noexpand\tempa{\temp}}}
\tempa
\edef\sfdname{\csname sfd@\tempa\endcsname}
}

\@namedef{sfd@bg5}{UBig5}
\@namedef{sfd@bg5+}{UBg5plus}
\@namedef{sfd@gb}{UGB}
\@namedef{sfd@gbk}{UGBK}
\@namedef{sfd@jis}{UJIS}
\@namedef{sfd@ks}{UKS}
\@namedef{sfd@utf8}{Unicode}


\def\makecmapfiles$1{
\CJKenc{$1}
\edef\lower@@@enc{\lowercase{\def\noexpand\lower@@@enc{\csname CJK@$1@nfssenc\endcsname}}}
\lower@@@enc
\get@sfd{$1}
\expandafter\ifx\csname \sfdname cmap\endcsname\relax
   \make@eks@active
   \input{\sfdname.sfd}
\else
   \csname \sfdname cmap\endcsname
\fi
}

\makecmapfiles{\cmapEnc}



\catcode`\x=11

\ifnum\the\cmaplines<256
   \ifnum\the\cmap@num@b > 0
      \write@cmapheader
      \write@cmaptotal
   \fi

   \cmap@num@a 0
   \cmap@num@b \cmaplines

   \ifnum\the\cmap@num@b > 100
      \cmap@num@b 100
      \write@cmapcontents
      \cmap@num@a 100
      \cmap@num@b \cmaplines
   \fi

   \ifnum\the\cmap@num@b > 200
      \cmap@num@b 200
      \write@cmapcontents
      \cmap@num@a 200
      \cmap@num@b \cmaplines
   \fi

   \ifnum\the\cmap@num@b > 0
      \write@cmapcontents
      \write@cmapend
   \fi
\fi





\end{document}

%
%       Then you will get 35 or 36 cmap files like c1001.cmap, c1002.cmap, ...
%
%
\documentclass{article}
\usepackage{CJK}
\begin{document}
\makeatletter

\newwrite\cmapout
\newif\if@firstplane
\@firstplanetrue

\newif\if@CJKplane@hexmode
\@CJKplane@hexmodefalse

\newcount\unicode@a
\newcount\unicode@b

\def\unicodecmap{
\@CJKplane@hexmodetrue
\unicode@a 0
\loop
  \CJK@numbToHex{\tempba}{\unicode@a}
  \ifnum\the\unicode@a < 256
     {\output@unicodeplane}
  \advance\unicode@a 1
\repeat
}

\def\output@unicodeplane
{{
\unicode@b 0
\loop
  \ifnum\the\unicode@b < 256
     \CJK@numbToHex{\tempbb}{\unicode@b}
     \edef\tempbb{\tempba\tempbb}
     \expandafter\@@cidmap\tempbb
  \advance\unicode@b 1
\repeat
}
}




\newcount\cmaplines
\newcount\cmapCJKplane
\newcount\cmaptemp
\newcount\cmap@num@a
\newcount\cmap@num@b

\cmapCJKplane 0
\cmaplines 0

\catcode`\%=11
\catcode`\#=14
\catcode`\$=6

\catcode`\x=13
\def x{\@@cidmap}

\def\make@eks@active{
    \catcode`\x=13
    \def x{\@@cidmap}}

\catcode`\x=11

\def\write@cmapheader{
   \begingroup
      \if@filesw
        \immediate\openout \cmapout  \lower@@@enc\cmap@CJKplane.cmap\relax
      \fi
      \@nobreakfalse
      \typeout{...generating \lower@@@enc\cmap@CJKplane.cmap...}
   \endgroup

   \immediate\write\cmapout{%!PS-Adobe-3.0 Resource-CMap}
   \immediate\write\cmapout{%%DocumentNeededResources: ProcSet (CIDInit)}
   \immediate\write\cmapout{%%IncludeResource: ProcSet (CIDInit)}
   \immediate\write\cmapout{%%BeginResource: CMap (TeX-\lower@@@enc-\cmap@CJKplane-0)}
   \immediate\write\cmapout{%%Title: (TeX-\lower@@@enc-\cmap@CJKplane-0\space
       TeX \lower@@@enc-\cmap@CJKplane 0)}
   \immediate\write\cmapout{%%Version: 1.000}
   \immediate\write\cmapout{%%EndComments}
   \immediate\write\cmapout{/CIDInit /ProcSet findresource begin}
   \immediate\write\cmapout{begincmap}
   \immediate\write\cmapout{12 dict begin}
   \immediate\write\cmapout{/CIDSystemInfo}
   \immediate\write\cmapout{<< /Registry (TeX)}
   \immediate\write\cmapout{/Ordering (\lower@@@enc-\cmap@CJKplane)}
   \immediate\write\cmapout{/Supplement 0}
   \immediate\write\cmapout{>> def}
   \immediate\write\cmapout{/CMapName /TeX-\lower@@@enc-\cmap@CJKplane-0 def}
   \immediate\write\cmapout{/CMapType 2 def}
}

\def\write@cmaptotal{
   \cmaptemp\cmaplines
   \advance\cmaptemp -1
   \CJK@numbToHex{\tempba}{\cmaptemp}
   \immediate\write\cmapout{1 begincodespacerange}
   \immediate\write\cmapout{<00> <\tempba>}
   \immediate\write\cmapout{endcodespacerange}
}

\def\write@cmapcontents{

   \cmaptemp\cmap@num@b
   \advance\cmaptemp -\cmap@num@a
   \immediate\write\cmapout{\the\cmaptemp\space beginbfchar}

   {\cmaptemp \cmap@num@a
   \loop
      \ifnum\the\cmaptemp<\cmap@num@b
      \immediate\write\cmapout{\csname cmapline@\the\cmaptemp\endcsname}
      \advance\cmaptemp 1
   \repeat
   \immediate\write\cmapout{endbfchar}
   }
}

\def\write@cmapend{
   \immediate\write\cmapout{endcmap}
   \immediate\write\cmapout{CMapName currentdict /CMap defineresource pop}
   \immediate\write\cmapout{end}
   \immediate\write\cmapout{end}
   \immediate\write\cmapout{%%EndResource}
   \immediate\write\cmapout{%%EOF}
   \immediate\closeout\cmapout
}

\def\@@cidmap$1$2$3$4{

\catcode`\x=11

\ifnum\the\cmaplines=0
   \if@firstplane
      \global\@firstplanefalse
      \ifnum $1=0
      \else
          \global\advance \cmapCJKplane 1
      \fi
   \else
          \global\advance \cmapCJKplane 1
   \fi
   \if@CJKplane@hexmode
      \CJK@numbToHex{\cmap@CJKplane}{\cmapCJKplane}
      \edef\tempa{\lowercase{\def\noexpand\cmap@CJKplane{\cmap@CJKplane}}}
      \tempa
   \else
      \ifnum\the\cmapCJKplane<10
            \edef\cmap@CJKplane{0\the\cmapCJKplane}
      \else
            \edef\cmap@CJKplane{\the\cmapCJKplane}
      \fi
   \fi
\fi
\CJK@numbToHex{\cmap@hexind}{\cmaplines}
\expandafter\edef\csname cmapline@\the\cmaplines\endcsname{<\cmap@hexind> <$1$2$3$4>}
\advance\cmaplines 1

\ifnum\the\cmaplines=256

   \write@cmapheader
   \write@cmaptotal

   \cmap@num@a 0
   \cmap@num@b 100
   \write@cmapcontents

   \cmap@num@a 100
   \cmap@num@b 200
   \write@cmapcontents

   \cmap@num@a 200
   \cmap@num@b 256
   \write@cmapcontents

   \write@cmapend
   \global\cmaplines 0
\fi

\make@eks@active
}


\def\get@sfd$1{
\edef\temp{$1}
\edef\tempa{\lowercase{\def\noexpand\tempa{\temp}}}
\tempa
\edef\sfdname{\csname sfd@\tempa\endcsname}
}

\@namedef{sfd@bg5}{UBig5}
\@namedef{sfd@bg5+}{UBg5plus}
\@namedef{sfd@gb}{UGB}
\@namedef{sfd@gbk}{UGBK}
\@namedef{sfd@jis}{UJIS}
\@namedef{sfd@ks}{UKS}
\@namedef{sfd@utf8}{Unicode}


\def\makecmapfiles$1{
\CJKenc{$1}
\edef\lower@@@enc{\lowercase{\def\noexpand\lower@@@enc{\csname CJK@$1@nfssenc\endcsname}}}
\lower@@@enc
\get@sfd{$1}
\expandafter\ifx\csname \sfdname cmap\endcsname\relax
   \make@eks@active
   \input{\sfdname.sfd}
\else
   \csname \sfdname cmap\endcsname
\fi
}

\makecmapfiles{\cmapEnc}



\catcode`\x=11

\ifnum\the\cmaplines<256
   \ifnum\the\cmap@num@b > 0
      \write@cmapheader
      \write@cmaptotal
   \fi

   \cmap@num@a 0
   \cmap@num@b \cmaplines

   \ifnum\the\cmap@num@b > 100
      \cmap@num@b 100
      \write@cmapcontents
      \cmap@num@a 100
      \cmap@num@b \cmaplines
   \fi

   \ifnum\the\cmap@num@b > 200
      \cmap@num@b 200
      \write@cmapcontents
      \cmap@num@a 200
      \cmap@num@b \cmaplines
   \fi

   \ifnum\the\cmap@num@b > 0
      \write@cmapcontents
      \write@cmapend
   \fi
\fi





\end{document}

%
%       Then you will get 35 or 36 cmap files like c1001.cmap, c1002.cmap, ...
%
%
\documentclass{article}
\usepackage{CJK}
\begin{document}
\makeatletter

\newwrite\cmapout
\newif\if@firstplane
\@firstplanetrue

\newif\if@CJKplane@hexmode
\@CJKplane@hexmodefalse

\newcount\unicode@a
\newcount\unicode@b

\def\unicodecmap{
\@CJKplane@hexmodetrue
\unicode@a 0
\loop
  \CJK@numbToHex{\tempba}{\unicode@a}
  \ifnum\the\unicode@a < 256
     {\output@unicodeplane}
  \advance\unicode@a 1
\repeat
}

\def\output@unicodeplane
{{
\unicode@b 0
\loop
  \ifnum\the\unicode@b < 256
     \CJK@numbToHex{\tempbb}{\unicode@b}
     \edef\tempbb{\tempba\tempbb}
     \expandafter\@@cidmap\tempbb
  \advance\unicode@b 1
\repeat
}
}




\newcount\cmaplines
\newcount\cmapCJKplane
\newcount\cmaptemp
\newcount\cmap@num@a
\newcount\cmap@num@b

\cmapCJKplane 0
\cmaplines 0

\catcode`\%=11
\catcode`\#=14
\catcode`\$=6

\catcode`\x=13
\def x{\@@cidmap}

\def\make@eks@active{
    \catcode`\x=13
    \def x{\@@cidmap}}

\catcode`\x=11

\def\write@cmapheader{
   \begingroup
      \if@filesw
        \immediate\openout \cmapout  \lower@@@enc\cmap@CJKplane.cmap\relax
      \fi
      \@nobreakfalse
      \typeout{...generating \lower@@@enc\cmap@CJKplane.cmap...}
   \endgroup

   \immediate\write\cmapout{%!PS-Adobe-3.0 Resource-CMap}
   \immediate\write\cmapout{%%DocumentNeededResources: ProcSet (CIDInit)}
   \immediate\write\cmapout{%%IncludeResource: ProcSet (CIDInit)}
   \immediate\write\cmapout{%%BeginResource: CMap (TeX-\lower@@@enc-\cmap@CJKplane-0)}
   \immediate\write\cmapout{%%Title: (TeX-\lower@@@enc-\cmap@CJKplane-0\space
       TeX \lower@@@enc-\cmap@CJKplane 0)}
   \immediate\write\cmapout{%%Version: 1.000}
   \immediate\write\cmapout{%%EndComments}
   \immediate\write\cmapout{/CIDInit /ProcSet findresource begin}
   \immediate\write\cmapout{begincmap}
   \immediate\write\cmapout{12 dict begin}
   \immediate\write\cmapout{/CIDSystemInfo}
   \immediate\write\cmapout{<< /Registry (TeX)}
   \immediate\write\cmapout{/Ordering (\lower@@@enc-\cmap@CJKplane)}
   \immediate\write\cmapout{/Supplement 0}
   \immediate\write\cmapout{>> def}
   \immediate\write\cmapout{/CMapName /TeX-\lower@@@enc-\cmap@CJKplane-0 def}
   \immediate\write\cmapout{/CMapType 2 def}
}

\def\write@cmaptotal{
   \cmaptemp\cmaplines
   \advance\cmaptemp -1
   \CJK@numbToHex{\tempba}{\cmaptemp}
   \immediate\write\cmapout{1 begincodespacerange}
   \immediate\write\cmapout{<00> <\tempba>}
   \immediate\write\cmapout{endcodespacerange}
}

\def\write@cmapcontents{

   \cmaptemp\cmap@num@b
   \advance\cmaptemp -\cmap@num@a
   \immediate\write\cmapout{\the\cmaptemp\space beginbfchar}

   {\cmaptemp \cmap@num@a
   \loop
      \ifnum\the\cmaptemp<\cmap@num@b
      \immediate\write\cmapout{\csname cmapline@\the\cmaptemp\endcsname}
      \advance\cmaptemp 1
   \repeat
   \immediate\write\cmapout{endbfchar}
   }
}

\def\write@cmapend{
   \immediate\write\cmapout{endcmap}
   \immediate\write\cmapout{CMapName currentdict /CMap defineresource pop}
   \immediate\write\cmapout{end}
   \immediate\write\cmapout{end}
   \immediate\write\cmapout{%%EndResource}
   \immediate\write\cmapout{%%EOF}
   \immediate\closeout\cmapout
}

\def\@@cidmap$1$2$3$4{

\catcode`\x=11

\ifnum\the\cmaplines=0
   \if@firstplane
      \global\@firstplanefalse
      \ifnum $1=0
      \else
          \global\advance \cmapCJKplane 1
      \fi
   \else
          \global\advance \cmapCJKplane 1
   \fi
   \if@CJKplane@hexmode
      \CJK@numbToHex{\cmap@CJKplane}{\cmapCJKplane}
      \edef\tempa{\lowercase{\def\noexpand\cmap@CJKplane{\cmap@CJKplane}}}
      \tempa
   \else
      \ifnum\the\cmapCJKplane<10
            \edef\cmap@CJKplane{0\the\cmapCJKplane}
      \else
            \edef\cmap@CJKplane{\the\cmapCJKplane}
      \fi
   \fi
\fi
\CJK@numbToHex{\cmap@hexind}{\cmaplines}
\expandafter\edef\csname cmapline@\the\cmaplines\endcsname{<\cmap@hexind> <$1$2$3$4>}
\advance\cmaplines 1

\ifnum\the\cmaplines=256

   \write@cmapheader
   \write@cmaptotal

   \cmap@num@a 0
   \cmap@num@b 100
   \write@cmapcontents

   \cmap@num@a 100
   \cmap@num@b 200
   \write@cmapcontents

   \cmap@num@a 200
   \cmap@num@b 256
   \write@cmapcontents

   \write@cmapend
   \global\cmaplines 0
\fi

\make@eks@active
}


\def\get@sfd$1{
\edef\temp{$1}
\edef\tempa{\lowercase{\def\noexpand\tempa{\temp}}}
\tempa
\edef\sfdname{\csname sfd@\tempa\endcsname}
}

\@namedef{sfd@bg5}{UBig5}
\@namedef{sfd@bg5+}{UBg5plus}
\@namedef{sfd@gb}{UGB}
\@namedef{sfd@gbk}{UGBK}
\@namedef{sfd@jis}{UJIS}
\@namedef{sfd@ks}{UKS}
\@namedef{sfd@utf8}{Unicode}


\def\makecmapfiles$1{
\CJKenc{$1}
\edef\lower@@@enc{\lowercase{\def\noexpand\lower@@@enc{\csname CJK@$1@nfssenc\endcsname}}}
\lower@@@enc
\get@sfd{$1}
\expandafter\ifx\csname \sfdname cmap\endcsname\relax
   \make@eks@active
   \input{\sfdname.sfd}
\else
   \csname \sfdname cmap\endcsname
\fi
}

\makecmapfiles{\cmapEnc}



\catcode`\x=11

\ifnum\the\cmaplines<256
   \ifnum\the\cmap@num@b > 0
      \write@cmapheader
      \write@cmaptotal
   \fi

   \cmap@num@a 0
   \cmap@num@b \cmaplines

   \ifnum\the\cmap@num@b > 100
      \cmap@num@b 100
      \write@cmapcontents
      \cmap@num@a 100
      \cmap@num@b \cmaplines
   \fi

   \ifnum\the\cmap@num@b > 200
      \cmap@num@b 200
      \write@cmapcontents
      \cmap@num@a 200
      \cmap@num@b \cmaplines
   \fi

   \ifnum\the\cmap@num@b > 0
      \write@cmapcontents
      \write@cmapend
   \fi
\fi





\end{document}
