% \iffalse meta-comment
%
% Copyright 1993 1994 1995 1996 1997 1998 1999 2000 2001 2002 2003 2004 2005 2006 2007 2008 2009
% The LaTeX3 Project and any individual authors listed elsewhere
% in this file. 
% 
% This file is part of the LaTeX base system.
% -------------------------------------------
% 
% It may be distributed and/or modified under the
% conditions of the LaTeX Project Public License, either version 1.3c
% of this license or (at your option) any later version.
% The latest version of this license is in
%    http://www.latex-project.org/lppl.txt
% and version 1.3c or later is part of all distributions of LaTeX 
% version 2005/12/01 or later.
% 
% This file has the LPPL maintenance status "maintained".
% 
% The list of all files belonging to the LaTeX base distribution is
% given in the file `manifest.txt'. See also `legal.txt' for additional
% information.
% 
% The list of derived (unpacked) files belonging to the distribution 
% and covered by LPPL is defined by the unpacking scripts (with 
% extension .ins) which are part of the distribution.
% 
% \fi
%%% ====================================================================
%%%  @LaTeX-file{
%%%     author          = "David Carlisle",
%%%     version         = "$Revision: 5652 $",
%%%     date            = "$Date: 2005-12-08 00:11:59 +0100 (Thu, 08 Dec 2005) $",
%%%     filename        = "latexbug.tex",
%%%     email           = "latex-bugs@latex-project.org",
%%%     codetable       = "ISO/ASCII",
%%%     keywords        = "LaTeX, bugs, reporting",
%%%     supported       = "yes",
%%%     docstring       = "
%%%
%%%     LaTeX bug report generator.
%%%     %%%%%%%%%%%%%%%%%%%%%%%%%%
%%%
%%%     Processing this file with LaTeX should produce
%%%     a file latexbug.msg in the current directory.
%%%
%%%     latexbug.msg may be used as a template for submitting bug
%%%     reports concerning files in the standard LaTeX distribution.
%%%
%%%     * Any bug report should include a small LaTeX test file
%%%       that shows the bug, and the log that LaTeX produces on the
%%%       test file.
%%%
%%%     * Reports should be in English.
%%%
%%%     * Reports should be as short as possible.
%%%
%%%     Please check before submitting a bug report that your format
%%%     is not more than two years old. New LaTeX releases occur at
%%%     regular intervals, and so your bug may be fixed in a later
%%%     release.
%%%
%%%     Completed bug report forms should be submitted to:
%%%     latex-bugs@latex-project.org
%%%
%%%     Please:
%%%     use the synopsis text as the `Subject' line of the message.
%%%     ===========================================================
%%%
%%%     For example:
%%%     Subject: \verb does not work inside \caption
%%%
%%%     So that your message has an identifying subject. 
%%%     Do not use subject lines such as `LaTeX bug' as this does not
%%%     help us to identify your message.
%%%
%%%     To follow up an existing report, include the bug reference, e.g.
%%%     "latex/1234: ", preceding the subject text.
%%%
%%%     Configuring latexbug
%%%     ====================
%%%     If you often test early release of LaTeX packages, and feel that
%%%     you may need to use this program often, you may create a file
%%%     latexbug.cfg
%%%     This should contain the answers of certain standard questions.
%%%     (Such as your name and email address.)
%%%
%%%     Currently the responses that may be stored in latexbug.cfg are:
%%%     name address organisation and interactive.
%%%     (If organisation is not set in the .cfg file latexbug does not
%%%     prompt for it, as this is not vital information.)
%%%
%%%     For example, a personal latexbug.cfg could look like
%%%     
%%%     \def\name{My Real Name}
%%%     \def\address{name@some.place}
%%%     \def\organisation{My Institution}
%%%     \def\interactive{y}
%%%     
%%%     A site latexbug.cfg might just set the \organisation, leaving
%%%     the personal details to be filled in interactively by the user.
%%%     
%%%  }
%%% ====================================================================

% Test if we are using INITEX (or a strange format that does not
% have { as a `group open' character).
\ifcat{=

\catcode`\{=1 \let\bgroup{
\catcode`\}=2 \let\egroup}
\catcode`\#=6 
\catcode`\^=7
\catcode`\@=11

\newlinechar`\^^J
\def\m@ne{-1 }
\countdef\count@255

\def\fmtname{INITEX}
\def\fmtversion{9999/00/00}
\def\@secondoftwo#1#2{#2}
\def\@empty{}
\everyjob{\typeout{INITEX}}
\def\space{ }
\def\@spaces{\space\space\space\space}
\let\@@end\end
\let\loop\relax

% hide outer macro names as we are inside an if
\expandafter\let\csname repeat\expandafter\endcsname
                \csname fi\endcsname

\chardef\msg15
\immediate\openout\msg=\jobname.msg

\expandafter\edef\csname newif\endcsname#1#2{%
  \let\noexpand\ifinteractive
    \expandafter\noexpand\csname iffalse\endcsname}

\expandafter\def\csname newcount\endcsname#1{}

\def\dospecials{\catcode`\\=12 }

\def\typeout{\immediate\write17}

\def\two@digits#1{\ifnum#1<10 0\fi\number#1}

\def\wmsg#1#{\bgroup\@wmsg}

\def\@ifundefined#1#2#3%
  {\expandafter\ifx\csname#1\endcsname\relax#2\else#3\fi}

\def\@inputfiles{INITEX}
\else
%%
%% @ is a letter
%%
\catcode`\@=11

%%
%% Grab the initex file list
%%
%% If this file is called via 
%%     latex "% \iffalse meta-comment
%
% Copyright 1993 1994 1995 1996 1997 1998 1999 2000 2001 2002 2003 2004 2005 2006 2007 2008 2009
% The LaTeX3 Project and any individual authors listed elsewhere
% in this file. 
% 
% This file is part of the LaTeX base system.
% -------------------------------------------
% 
% It may be distributed and/or modified under the
% conditions of the LaTeX Project Public License, either version 1.3c
% of this license or (at your option) any later version.
% The latest version of this license is in
%    http://www.latex-project.org/lppl.txt
% and version 1.3c or later is part of all distributions of LaTeX 
% version 2005/12/01 or later.
% 
% This file has the LPPL maintenance status "maintained".
% 
% The list of all files belonging to the LaTeX base distribution is
% given in the file `manifest.txt'. See also `legal.txt' for additional
% information.
% 
% The list of derived (unpacked) files belonging to the distribution 
% and covered by LPPL is defined by the unpacking scripts (with 
% extension .ins) which are part of the distribution.
% 
% \fi
%%% ====================================================================
%%%  @LaTeX-file{
%%%     author          = "David Carlisle",
%%%     version         = "$Revision: 5652 $",
%%%     date            = "$Date: 2005-12-08 00:11:59 +0100 (Thu, 08 Dec 2005) $",
%%%     filename        = "latexbug.tex",
%%%     email           = "latex-bugs@latex-project.org",
%%%     codetable       = "ISO/ASCII",
%%%     keywords        = "LaTeX, bugs, reporting",
%%%     supported       = "yes",
%%%     docstring       = "
%%%
%%%     LaTeX bug report generator.
%%%     %%%%%%%%%%%%%%%%%%%%%%%%%%
%%%
%%%     Processing this file with LaTeX should produce
%%%     a file latexbug.msg in the current directory.
%%%
%%%     latexbug.msg may be used as a template for submitting bug
%%%     reports concerning files in the standard LaTeX distribution.
%%%
%%%     * Any bug report should include a small LaTeX test file
%%%       that shows the bug, and the log that LaTeX produces on the
%%%       test file.
%%%
%%%     * Reports should be in English.
%%%
%%%     * Reports should be as short as possible.
%%%
%%%     Please check before submitting a bug report that your format
%%%     is not more than two years old. New LaTeX releases occur at
%%%     regular intervals, and so your bug may be fixed in a later
%%%     release.
%%%
%%%     Completed bug report forms should be submitted to:
%%%     latex-bugs@latex-project.org
%%%
%%%     Please:
%%%     use the synopsis text as the `Subject' line of the message.
%%%     ===========================================================
%%%
%%%     For example:
%%%     Subject: \verb does not work inside \caption
%%%
%%%     So that your message has an identifying subject. 
%%%     Do not use subject lines such as `LaTeX bug' as this does not
%%%     help us to identify your message.
%%%
%%%     To follow up an existing report, include the bug reference, e.g.
%%%     "latex/1234: ", preceding the subject text.
%%%
%%%     Configuring latexbug
%%%     ====================
%%%     If you often test early release of LaTeX packages, and feel that
%%%     you may need to use this program often, you may create a file
%%%     latexbug.cfg
%%%     This should contain the answers of certain standard questions.
%%%     (Such as your name and email address.)
%%%
%%%     Currently the responses that may be stored in latexbug.cfg are:
%%%     name address organisation and interactive.
%%%     (If organisation is not set in the .cfg file latexbug does not
%%%     prompt for it, as this is not vital information.)
%%%
%%%     For example, a personal latexbug.cfg could look like
%%%     
%%%     \def\name{My Real Name}
%%%     \def\address{name@some.place}
%%%     \def\organisation{My Institution}
%%%     \def\interactive{y}
%%%     
%%%     A site latexbug.cfg might just set the \organisation, leaving
%%%     the personal details to be filled in interactively by the user.
%%%     
%%%  }
%%% ====================================================================

% Test if we are using INITEX (or a strange format that does not
% have { as a `group open' character).
\ifcat{=

\catcode`\{=1 \let\bgroup{
\catcode`\}=2 \let\egroup}
\catcode`\#=6 
\catcode`\^=7
\catcode`\@=11

\newlinechar`\^^J
\def\m@ne{-1 }
\countdef\count@255

\def\fmtname{INITEX}
\def\fmtversion{9999/00/00}
\def\@secondoftwo#1#2{#2}
\def\@empty{}
\everyjob{\typeout{INITEX}}
\def\space{ }
\def\@spaces{\space\space\space\space}
\let\@@end\end
\let\loop\relax

% hide outer macro names as we are inside an if
\expandafter\let\csname repeat\expandafter\endcsname
                \csname fi\endcsname

\chardef\msg15
\immediate\openout\msg=\jobname.msg

\expandafter\edef\csname newif\endcsname#1#2{%
  \let\noexpand\ifinteractive
    \expandafter\noexpand\csname iffalse\endcsname}

\expandafter\def\csname newcount\endcsname#1{}

\def\dospecials{\catcode`\\=12 }

\def\typeout{\immediate\write17}

\def\two@digits#1{\ifnum#1<10 0\fi\number#1}

\def\wmsg#1#{\bgroup\@wmsg}

\def\@ifundefined#1#2#3%
  {\expandafter\ifx\csname#1\endcsname\relax#2\else#3\fi}

\def\@inputfiles{INITEX}
\else
%%
%% @ is a letter
%%
\catcode`\@=11

%%
%% Grab the initex file list
%%
%% If this file is called via 
%%     latex "% \iffalse meta-comment
%
% Copyright 1993 1994 1995 1996 1997 1998 1999 2000 2001 2002 2003 2004 2005 2006 2007 2008 2009
% The LaTeX3 Project and any individual authors listed elsewhere
% in this file. 
% 
% This file is part of the LaTeX base system.
% -------------------------------------------
% 
% It may be distributed and/or modified under the
% conditions of the LaTeX Project Public License, either version 1.3c
% of this license or (at your option) any later version.
% The latest version of this license is in
%    http://www.latex-project.org/lppl.txt
% and version 1.3c or later is part of all distributions of LaTeX 
% version 2005/12/01 or later.
% 
% This file has the LPPL maintenance status "maintained".
% 
% The list of all files belonging to the LaTeX base distribution is
% given in the file `manifest.txt'. See also `legal.txt' for additional
% information.
% 
% The list of derived (unpacked) files belonging to the distribution 
% and covered by LPPL is defined by the unpacking scripts (with 
% extension .ins) which are part of the distribution.
% 
% \fi
%%% ====================================================================
%%%  @LaTeX-file{
%%%     author          = "David Carlisle",
%%%     version         = "$Revision: 5652 $",
%%%     date            = "$Date: 2005-12-08 00:11:59 +0100 (Thu, 08 Dec 2005) $",
%%%     filename        = "latexbug.tex",
%%%     email           = "latex-bugs@latex-project.org",
%%%     codetable       = "ISO/ASCII",
%%%     keywords        = "LaTeX, bugs, reporting",
%%%     supported       = "yes",
%%%     docstring       = "
%%%
%%%     LaTeX bug report generator.
%%%     %%%%%%%%%%%%%%%%%%%%%%%%%%
%%%
%%%     Processing this file with LaTeX should produce
%%%     a file latexbug.msg in the current directory.
%%%
%%%     latexbug.msg may be used as a template for submitting bug
%%%     reports concerning files in the standard LaTeX distribution.
%%%
%%%     * Any bug report should include a small LaTeX test file
%%%       that shows the bug, and the log that LaTeX produces on the
%%%       test file.
%%%
%%%     * Reports should be in English.
%%%
%%%     * Reports should be as short as possible.
%%%
%%%     Please check before submitting a bug report that your format
%%%     is not more than two years old. New LaTeX releases occur at
%%%     regular intervals, and so your bug may be fixed in a later
%%%     release.
%%%
%%%     Completed bug report forms should be submitted to:
%%%     latex-bugs@latex-project.org
%%%
%%%     Please:
%%%     use the synopsis text as the `Subject' line of the message.
%%%     ===========================================================
%%%
%%%     For example:
%%%     Subject: \verb does not work inside \caption
%%%
%%%     So that your message has an identifying subject. 
%%%     Do not use subject lines such as `LaTeX bug' as this does not
%%%     help us to identify your message.
%%%
%%%     To follow up an existing report, include the bug reference, e.g.
%%%     "latex/1234: ", preceding the subject text.
%%%
%%%     Configuring latexbug
%%%     ====================
%%%     If you often test early release of LaTeX packages, and feel that
%%%     you may need to use this program often, you may create a file
%%%     latexbug.cfg
%%%     This should contain the answers of certain standard questions.
%%%     (Such as your name and email address.)
%%%
%%%     Currently the responses that may be stored in latexbug.cfg are:
%%%     name address organisation and interactive.
%%%     (If organisation is not set in the .cfg file latexbug does not
%%%     prompt for it, as this is not vital information.)
%%%
%%%     For example, a personal latexbug.cfg could look like
%%%     
%%%     \def\name{My Real Name}
%%%     \def\address{name@some.place}
%%%     \def\organisation{My Institution}
%%%     \def\interactive{y}
%%%     
%%%     A site latexbug.cfg might just set the \organisation, leaving
%%%     the personal details to be filled in interactively by the user.
%%%     
%%%  }
%%% ====================================================================

% Test if we are using INITEX (or a strange format that does not
% have { as a `group open' character).
\ifcat{=

\catcode`\{=1 \let\bgroup{
\catcode`\}=2 \let\egroup}
\catcode`\#=6 
\catcode`\^=7
\catcode`\@=11

\newlinechar`\^^J
\def\m@ne{-1 }
\countdef\count@255

\def\fmtname{INITEX}
\def\fmtversion{9999/00/00}
\def\@secondoftwo#1#2{#2}
\def\@empty{}
\everyjob{\typeout{INITEX}}
\def\space{ }
\def\@spaces{\space\space\space\space}
\let\@@end\end
\let\loop\relax

% hide outer macro names as we are inside an if
\expandafter\let\csname repeat\expandafter\endcsname
                \csname fi\endcsname

\chardef\msg15
\immediate\openout\msg=\jobname.msg

\expandafter\edef\csname newif\endcsname#1#2{%
  \let\noexpand\ifinteractive
    \expandafter\noexpand\csname iffalse\endcsname}

\expandafter\def\csname newcount\endcsname#1{}

\def\dospecials{\catcode`\\=12 }

\def\typeout{\immediate\write17}

\def\two@digits#1{\ifnum#1<10 0\fi\number#1}

\def\wmsg#1#{\bgroup\@wmsg}

\def\@ifundefined#1#2#3%
  {\expandafter\ifx\csname#1\endcsname\relax#2\else#3\fi}

\def\@inputfiles{INITEX}
\else
%%
%% @ is a letter
%%
\catcode`\@=11

%%
%% Grab the initex file list
%%
%% If this file is called via 
%%     latex "% \iffalse meta-comment
%
% Copyright 1993 1994 1995 1996 1997 1998 1999 2000 2001 2002 2003 2004 2005 2006 2007 2008 2009
% The LaTeX3 Project and any individual authors listed elsewhere
% in this file. 
% 
% This file is part of the LaTeX base system.
% -------------------------------------------
% 
% It may be distributed and/or modified under the
% conditions of the LaTeX Project Public License, either version 1.3c
% of this license or (at your option) any later version.
% The latest version of this license is in
%    http://www.latex-project.org/lppl.txt
% and version 1.3c or later is part of all distributions of LaTeX 
% version 2005/12/01 or later.
% 
% This file has the LPPL maintenance status "maintained".
% 
% The list of all files belonging to the LaTeX base distribution is
% given in the file `manifest.txt'. See also `legal.txt' for additional
% information.
% 
% The list of derived (unpacked) files belonging to the distribution 
% and covered by LPPL is defined by the unpacking scripts (with 
% extension .ins) which are part of the distribution.
% 
% \fi
%%% ====================================================================
%%%  @LaTeX-file{
%%%     author          = "David Carlisle",
%%%     version         = "$Revision: 5652 $",
%%%     date            = "$Date: 2005-12-08 00:11:59 +0100 (Thu, 08 Dec 2005) $",
%%%     filename        = "latexbug.tex",
%%%     email           = "latex-bugs@latex-project.org",
%%%     codetable       = "ISO/ASCII",
%%%     keywords        = "LaTeX, bugs, reporting",
%%%     supported       = "yes",
%%%     docstring       = "
%%%
%%%     LaTeX bug report generator.
%%%     %%%%%%%%%%%%%%%%%%%%%%%%%%
%%%
%%%     Processing this file with LaTeX should produce
%%%     a file latexbug.msg in the current directory.
%%%
%%%     latexbug.msg may be used as a template for submitting bug
%%%     reports concerning files in the standard LaTeX distribution.
%%%
%%%     * Any bug report should include a small LaTeX test file
%%%       that shows the bug, and the log that LaTeX produces on the
%%%       test file.
%%%
%%%     * Reports should be in English.
%%%
%%%     * Reports should be as short as possible.
%%%
%%%     Please check before submitting a bug report that your format
%%%     is not more than two years old. New LaTeX releases occur at
%%%     regular intervals, and so your bug may be fixed in a later
%%%     release.
%%%
%%%     Completed bug report forms should be submitted to:
%%%     latex-bugs@latex-project.org
%%%
%%%     Please:
%%%     use the synopsis text as the `Subject' line of the message.
%%%     ===========================================================
%%%
%%%     For example:
%%%     Subject: \verb does not work inside \caption
%%%
%%%     So that your message has an identifying subject. 
%%%     Do not use subject lines such as `LaTeX bug' as this does not
%%%     help us to identify your message.
%%%
%%%     To follow up an existing report, include the bug reference, e.g.
%%%     "latex/1234: ", preceding the subject text.
%%%
%%%     Configuring latexbug
%%%     ====================
%%%     If you often test early release of LaTeX packages, and feel that
%%%     you may need to use this program often, you may create a file
%%%     latexbug.cfg
%%%     This should contain the answers of certain standard questions.
%%%     (Such as your name and email address.)
%%%
%%%     Currently the responses that may be stored in latexbug.cfg are:
%%%     name address organisation and interactive.
%%%     (If organisation is not set in the .cfg file latexbug does not
%%%     prompt for it, as this is not vital information.)
%%%
%%%     For example, a personal latexbug.cfg could look like
%%%     
%%%     \def\name{My Real Name}
%%%     \def\address{name@some.place}
%%%     \def\organisation{My Institution}
%%%     \def\interactive{y}
%%%     
%%%     A site latexbug.cfg might just set the \organisation, leaving
%%%     the personal details to be filled in interactively by the user.
%%%     
%%%  }
%%% ====================================================================

% Test if we are using INITEX (or a strange format that does not
% have { as a `group open' character).
\ifcat{=

\catcode`\{=1 \let\bgroup{
\catcode`\}=2 \let\egroup}
\catcode`\#=6 
\catcode`\^=7
\catcode`\@=11

\newlinechar`\^^J
\def\m@ne{-1 }
\countdef\count@255

\def\fmtname{INITEX}
\def\fmtversion{9999/00/00}
\def\@secondoftwo#1#2{#2}
\def\@empty{}
\everyjob{\typeout{INITEX}}
\def\space{ }
\def\@spaces{\space\space\space\space}
\let\@@end\end
\let\loop\relax

% hide outer macro names as we are inside an if
\expandafter\let\csname repeat\expandafter\endcsname
                \csname fi\endcsname

\chardef\msg15
\immediate\openout\msg=\jobname.msg

\expandafter\edef\csname newif\endcsname#1#2{%
  \let\noexpand\ifinteractive
    \expandafter\noexpand\csname iffalse\endcsname}

\expandafter\def\csname newcount\endcsname#1{}

\def\dospecials{\catcode`\\=12 }

\def\typeout{\immediate\write17}

\def\two@digits#1{\ifnum#1<10 0\fi\number#1}

\def\wmsg#1#{\bgroup\@wmsg}

\def\@ifundefined#1#2#3%
  {\expandafter\ifx\csname#1\endcsname\relax#2\else#3\fi}

\def\@inputfiles{INITEX}
\else
%%
%% @ is a letter
%%
\catcode`\@=11

%%
%% Grab the initex file list
%%
%% If this file is called via 
%%     latex "\input{latexbug}" or some
%% similar command sequence rather than 
%%     latex latexbug
%% then the debugging info in \reserved@a will already have been lost.
%% This might not matter, but if it does we may ask the user to resubmit
%% the report.
\ifx\reserved@b\@undefined
  \ifx\reserved@a\@gobble
    \def\@inputfiles{NONE}
  \else
    \let\@inputfiles\reserved@a
  \fi
\else
  \def\@inputfiles{LOST}
\fi

%%
%% Output stream to produce the bug report template.
%%
\newwrite\msg
\immediate\openout\msg=\jobname.msg


%%
%% Check that LaTeX2e is being used.
%%
\ifx\undefined\newcommand
 \newlinechar`\^^J%
 \immediate\write17{^^J%
    You must use LaTeX2e to generate the bug report!^^J^^J%
    If there is a bug in the installation procedure,^^J%
    and you can not create LaTeX2e, you may use initex^^J%
    to generate the report}%

 \let\relax\end
\else
\def\@tempa{LaTeX2e}\ifx\@tempa\fmtname\else
 \immediate\write17{^^J%
  Older Versions of LaTeX are no longer supported.^^J%
  You must use LaTeX2e to generate the bug report!^^J^^J%
  If there is a bug in the installation procedure,^^J%
  and you can not create LaTeX2e, you may use initex^^J%
  to generate the report}%
 \let\relax\@@end
\fi\fi

%%
%% \wmsg writes to the terminal, and the .msg file
%% \wmsg* just writes to the .msg file
%% \typeout just writes to the terminal
%%

\def\wmsg{\bgroup\@ifstar{\interactivefalse\@wmsg}\@wmsg}

\fi

\relax
\endlinechar=-1

\def\@wmsg#1{%
  \ifinteractive\immediate\write17{#1}\fi
  \immediate\write\msg{#1}%
  \egroup}

%%
%% if \interactivefalse just make a blank template.
%%
\newif\ifinteractive
\interactivetrue

%%
%% Prompt for an answer from the user, if the answer is not
%% provided by the cfg file.
%%

\def\readifnotknown#1{%
 \@ifundefined{#1}%
    {{\message{#1> }%
     \catcode`\^^I=12 \let\do\@makeother\dospecials
     \global\read\m@ne t\expandafter o\csname#1\endcsname}}%
    {\message{\csname#1\endcsname}}}

%%
%% Pause so messages do not scroll off screen.
%%
\def\pause{%
  \ifinteractive
    \message{Press <return> key to continue. }%
    \read\m@ne to \@tempa
  \fi}

%%
%% Opening Banner.
%%
\typeout{^^J%
============================================================^^J%
^^J%
LaTeX bug report generator^^J%
==========================^^J%
Processing this file with LaTeX will produce a template \jobname.msg^^J
for submitting bug reports for the LaTeX distribution.^^J^^J
* Please do not report bugs in contributed, non-standard, files to the
  ^^J \space latex-bugs address.^^J
* Please write your report in English.^^J
* Please keep the report as short as possible.^^J%
* If possible, check whether the bug has already been reported.^^J %
  \space The bugs database is available on WWW:^^J \space
   http://www.latex-project.org/bugs.html^^J}


\ifinteractive
  \InputIfFileExists{latexbug.cfg}{\typeout{** latexbug.cfg used **}}{}
\fi

%% \batch is a `private' macro used to get a batchmode
%% (actually \nonstopmode) run for use with latexbug.el
\ifx\batch\undefined

\def\getcategory{%
\count@=0
\ifinteractive

\typeout{%
* Please carefully select the category as different categories^^J
\space\space are supported by different people!^^J^^J%
0) LaTeX:\@spaces
         The `base' format, and standard classes only (base).^^J
1) tools:\@spaces
         Packages supported by the LaTeX3 project team (tools).^^J
2) graphics:\space 
         The color and graphics packages (graphics).^^J
3) mfnfss: \space\space
         --- Don't use, no longer supported! ---^^J
4) psnfss: \space\space
         Packages for using PostScript fonts LaTeX (psnfss).^^J
5) amslatex:\space
         AMS supported Classes and Packages (amsfonts and amslatex).^^J
6) babel:\@spaces
         Packages supporting many different languages (babel).^^J%
7) expl3:\@spaces
         Experimental packages for TeX programmers. (expl3)^^J%
8) cyrillic: Packages for using Cyrillic fonts (cyrillic).^^J}

\message{Please select a category 0--8:  }
\read\m@ne to \answer
\if!\answer!\let\answer\m@ne\fi
\count@=\answer\relax
\else
\typeout{As you are using INITEX, I will assume category `latex'}
\fi

\ifcase\count@
\def\category{latex}\or
\def\category{tools}\or
\def\category{graphics}\or
\errhelp{Retry with <return>, or quit with x.}
\def\badcategory{This category is no longer supported}
\errmessage{\badcategory}
\expandafter\getcategory
\or
\def\category{psnfss}\or
\def\category{amslatex}\or
\def\category{babel}\or
\def\category{expl3}\or
\def\category{cyrillic}%
\else
\errhelp{Retry with <return>, or quit with x.}
\def\badcategory{Only categories 0,...,8 are supported at this time}
\errmessage{\badcategory}
\expandafter\getcategory
\fi}
\getcategory


\typeout{^^J%
============================================================^^J%
^^J%
Please give a one line ( < 50 character ) description of the problem.%
^^J^^J%
If you are using email to report the problem,^^J%
please also use this text as the `Subject' line for the mail message:%
^^J \@spaces\@spaces\space
                 |<------------------------------------------------>|} 


\loop
\let\synopsis\relax
\readifnotknown{synopsis}
\ifx\synopsis\@empty
\repeat


\typeout{%
^^J%
\ifinteractive
This report generator may be used in one of two ways.^^J%
If you choose the interactive option, you will be prompted to answer^^J%
several questions. Otherwise a blank template will be created for^^J%
you to fill in using your editor.^^J%
\else
INITEX should only be used for reporting bugs with the LaTeX2e^^J%
installation procedure. If you have a working copy of LaTeX2e,^^J%
please use that to generate the report.
\fi}

\ifinteractive
\typeout{Interactive session (y/n) ? }
\readifnotknown{interactive}

\ifx\interactive\@empty
   \def\interactive{n}
\fi

%%
%% Allow anything begining with `y' or `Y' for yes.
%%
\edef\interactive{\uccode`\expandafter\@car\interactive\@nil}
\ifnum \interactive=`Y \else\interactivefalse\fi
\else
\def\interactive{`\N}
\fi

\else
\def\category{< CATEGORY >}
\def\synopsis{< SYNOPSIS >}
\batchmode
\interactivefalse
\def\interactive{`\N}
\fi

%%
%% Header in the msg file.
%%
\wmsg*{^^J%
 LaTeX2e bug report.^^J%
\ifnum \interactive=`Y Generated \else Template generated \fi
 by latexbug.tex on \number\year/\two@digits\month/\two@digits\day^^J%
^^J%
 Reports may be submitted by email to latex-bugs@latex-project.org^^J%
 Please use the subject line:^^J%
 Subject: \synopsis^^J%
^^J%
 To follow up an existing report, include the bug reference, e.g.^^J
 "latex/1234: ", preceding the subject text.^^J
 ============================================================^^J}



%%
%% Category of bug, obtained earlier but put out now, after the header.
%%
\wmsg{>Category: \category}

%%
%% synopsis of bug, obtained earlier but put out now, after the header.
%%
\wmsg{>Synopsis: \synopsis}


%%
%% >Confidential: Default to no unless this is overridden
%% in latexbug.cfg. If you want to send a one-off confidential
%% report, just edit the latexbug.msg file to say yes.
%%
\wmsg{>Confidential: \ifx\confidential\undefined
                         no
                       \else
                         \confidential
                       \fi}


%% Try to catch various formats that babel has used to add to the banner
%% over the years. Currently it uses \typeout, so put the first \typeout
%% in \format and the second in \hyphenation. Earlier releases used 
%% \immedite\write
\begingroup
 \global\let\format\@empty
 \gdef\hyphenation{standard}
 \def\immediate#1#{\xdef\hyphenation}
 \def\typeout#1{%
    \xdef\format{\format#1}\def\typeout##1{\xdef\hyphenation{##1}}}
 \the\everyjob
\endgroup

\wmsg{>Release: \format}

\ifinteractive
%%
%% if interactive, \wread reads a line (verbatim) and writes it to the
%% .msg file, until two consecutive blank lines are entered.
%%
  \def\wread{%
    \begingroup
    \catcode`\^^I=12 
    \let\do\@makeother\dospecials
    \wreadloop}

  \def\wreadloop{%
    \let\lastanswer\answer
    \message{=> }\read\m@ne to \answer
    \ifx\lastanswer\@empty
      \let\lastanswer\answer
    \fi
    \ifx\lastanswer\@empty
      \endgroup
    \else
      \immediate\write\msg{\answer}%
      \expandafter\wreadloop
    \fi}
\else
%%
%% If non-interactive, \wread just writes a blank line to the .msg file,
%% and \wmsg does not write to the terminal.
%%
  \def\wread{\wmsg{}}
\fi

%%
%% \copytomsg copies the contents of a file into the .msg file.
%% (at least it does it as well as TeX can, so there may be
%% transcription problems with 8-bit characters).
%%
%% It does a line count, and complains if the test file is
%% too large.

\chardef\inputfile=15

\newcount\linecount

\def\copytomsg#1{{%
   \def\do##1{\catcode`##1=11}%
   \dospecials
   \global\linecount\z@
   \openin\inputfile#1\relax
   \def\thefile{#1}%
   \@copytomsg
   \closein\inputfile}}

\def\@copytomsg{%
   \ifeof\inputfile
      \typeout{*** \thefile\space line count = \the\linecount}
   \else
      \global\advance\linecount\@ne
      \read\inputfile to \inputline
      \ifx\inputline\@empty
         \wmsg*{}
      \else
         \wmsg*{\inputline}
      \fi
      \expandafter\@copytomsg
   \fi}


%%
%% Test the age of the current format.
%%
\def\getage#1/#2/#3\@nil{%
  \count@\year
  \advance\count@-#1\relax
  \multiply\count@ by 12\relax
  \advance\count@\month
  \advance\count@-#2\relax}
%
\expandafter\getage\fmtversion\@nil
%%
%% \count@ should now be the age of the format in months.
%%
%% Take a generous definition of `year'.
\ifnum\count@>65
\def\oldformat{^^J%
   ! Your LaTeX installation is more than five years old.^^J%
   ! Please consider updating LaTeX before submitting this report.^^J%
   ! At least check a current LaTeX changes.txt file, to see if the^^J%
   ! bug has been fixed in the current release.^^J%
   !}
%%
%% Put the message in a macro to improve the look of the error mesage.
%%

\errhelp{If you still wish to complete the form, just type return.}
\errmessage{\oldformat}
\fi

%%
%% Test fonts not customised too much. (unless using initex)
%%
\ifx\loop\relax\else
\edef\fontdefaults{%
  \encodingdefault/\familydefault/\seriesdefault/\shapedefault}
\def\standardfontdefaults{OT1/cmr/m/n}
\ifx\fontdefaults\standardfontdefaults\else
\wmsg*{>Unformatted:}
\wmsg{Normal font: \fontdefaults}
\def\customisedfonts{^^J%
! This format has customised font defaults.^^J%
! Please try to re-create the error using a standard format^^J%
! before submitting this report}
\errhelp{If you still wish to complete the form, just type return.}
\errmessage{\customisedfonts}
\fi
\fi

%%
%% Now use \wmsg and \wread for each of the multi-line fields 
%% in the form.
%% One line fields use \readifnotknown, which will only prompt
%% if the field has not already been set in the configuration file.
%%
\ifinteractive
  \typeout{^^JYour name:}
  \readifnotknown{name}
\else
  \ifx\name\undefined
    \def\name{ < ENTER YOUR NAME > }
  \fi
\fi


\ifinteractive
  \typeout{^^JYour Address (preferably email):}
  \readifnotknown{address}
\else
  \ifx\address\undefined
    \def\address{ < ENTER YOUR EMAIL ADDRESS > }
  \fi
\fi

\wmsg*{>Originator: \address \space(\name)}

%%
%% >Organisation: is really a GNATS multiline field
%% but we treat it as a one-line field.
%%
\wmsg*{>Organization: \ifx\organisation\undefined
                        \ifx\organization\undefined\else
                           \organization
                        \fi
                       \else
                         \organisation
                       \fi}


%%
%% Test which format is being used. These fields are completed
%% automatically even if the blank template is being produced.
%%

\wmsg*{>Environment:}
\wmsg*{ Hyphenation: \hyphenation}
\wmsg*{ \string\@TeXversion: \meaning\@TeXversion
        \ifx\@TeXversion\@@undefined
         \space (Standard setting for TeX3.141 and later)\fi}
\wmsg*{ \string\@currdir: \meaning\@currdir}
\wmsg*{ \string\input@path: \meaning\input@path
        \ifx\input@path\@@undefined
         \space (Standard setting)\fi}

\wmsg*{>Unformatted:}
\wmsg*{ *** Initex configuration files}
%%
%% These are in a comma separated list, so locally reset
%% \newlinechar
{\newlinechar`\,
 \wmsg*{\@inputfiles}}
\wmsg*{ ***}

\wmsg*{>Description:}
\typeout{}
\wmsg{Description of bug:}
\ifinteractive
  \typeout{%
    \@spaces The answer to this  question may take several lines.^^J%
    \@spaces (Each such line will be prompted by =>.)^^J%
    \@spaces Typing TWO consecutive blank lines terminates the answer.}
\else
\wmsg{ < ENTER BUG REPORT HERE >}
\fi
\wread



%%
%% insertion of the test file
%%



\ifinteractive
   \typeout{^^J%
 Name of a SHORT, SELF-CONTAINED file which indicates the problem:^^J%
 This file should be as small as possible (preferably < 60 lines)^^J%
 Any non-standard files that the test file uses should be included^^J%
 using the filecontents environment.^^J^^J%
%
 LaTeX will try to input this file, so give the full path^^J%
 if the file is not in the current directory.^^J^^J%
%
 If you are not reporting a bug, and there is therefore^^J%
 no test file, just hit <return>}
   \message{filename> }\read\m@ne to \filename
\else
   \def\filename{}
\fi

%% 
%% Try to find the .tex file and .log file
%%


\ifx\filename\@empty
  \ifx\LaTeX\undefinedcommand
  \else
    \ifinteractive
      \typeout{^^J^^JNo test file.^^J^^J%
      Three classes of report are supported:^^J^^J%
      0) sw-bug:^^J\@spaces
         Bug in the software, the report should include a test file.^^J
      1) doc-bug:^^J\@spaces
         Inaccuracies in the documentation.^^J
      2) change-request:^^J\@spaces 
         Not a bug, but rather a request for LaTeX to be changed.^^J}
      \message{Please select a category 0--2:  }
      \read\m@ne to \answer
      \ifx\answer\@empty
        \def\answer{-1}
      \fi
      \count@=\answer\relax
      \else
        \count@=\z@
      \fi
    \ifcase\count@
      \ifinteractive\wmsg{>Class: sw-bug}\fi
      \typeout{^^J! Please edit the message to add a test file and log!}
      \pause
      \wmsg*{^^J>How-To-Repeat:}
      \wmsg*{%
      Sample file which indicates the problem:^^J%
      ========================================^^J%
      \space< TEST FILE HERE >^^J%
      ^^J%
      The log file from running LaTeX on the sample:^^J%
      ==============================================^^J%
      \space< LOG FROM TEST FILE HERE >}
    \or
      \wmsg{>Class: doc-bug}
    \or
      \wmsg{>Class: change-request}
    \else
      \errhelp{Quit with `x' and then re-start latexbug}
      \def\badcategory{Only classes 0,1,2 are supported at this time}
      \errmessage{\badcategory}
    \fi
  \fi
\else

\filename@parse\filename

\IfFileExists{\filename}{\edef\samplefile{\filename}}{}

\IfFileExists{\filename@area\filename@base.log}
  {\edef\logfile{\filename@area\filename@base.log}}
  {\IfFileExists{\filename@area\filename@base.lis}
    {\edef\logfile{\filename@area\filename@base.lis}}
    {}}


%% 
%% The example file goes here:
%%
\wmsg*{^^J>How-To-Repeat:}

\wmsg*{^^J%
Sample file which indicates the problem:^^J%
========================================}

\ifx\samplefile\undefinedcommand
   \typeout{^^J%
      Sample file \filename\space not found.^^J%
      Please edit \jobname.msg to include the sample file.}
   \wmsg*{ < TEST FILE HERE >}
   \pause
\else
   \copytomsg{\samplefile}
   \ifnum\linecount>60
    \typeout{%
^^J%
!!! Your test file is \the\linecount\space lines long.^^J%
!!! Such a large test file causes us problems:^^J%
!!! * It makes it difficult to track down the error^^J%
!!! * It makes our database for storing reports unnecessarily large.^^J%
!!! ^^J%
!!! Please, if at all possible, cut down your test file to the^^J%
!!! smallest file that shows the behaviour.^^J}
   \pause
   \fi
\fi


%%
%% The log file goes here:
%%
\wmsg*{^^J%
The log file from running LaTeX on the sample:^^J%
==============================================}

\ifx\logfile\undefinedcommand
   \typeout{^^J%
      Log file \filename@area\filename@base.log not found.^^J%
      Please edit \jobname.msg to include the log file.}
   \wmsg*{ < < LOG FROM TEST FILE HERE >}
   \pause
\else
   \copytomsg{\logfile}
\fi

\fi


%%
%% Closing Banner.
%%
\typeout{^^J%
============================================================}

\ifinteractive
 \typeout{^^J%
   You may wish to make further changes to the bug report file:^^J%
   `\jobname.msg'^^J%
   using your editor.}
\else
 \typeout{^^J%
   A template for submitting bug reports has been left in the file:^^J%
   \jobname.msg^^J%
   Please use your editor to complete the file before submitting^^J%
   your report.}
\fi

\let\ifinteractivetrue\iftrue
\typeout{^^J%
  If you have access to email, please send `\jobname.msg' to:^^J%
  latex-bugs@latex-project.org  Please use the subject line:^^J%
  \@spaces Subject: \synopsis^^J%
^^J%
 (This subject will be used in all subsequent correspondence.)^^J%
^^J%
 To follow up an existing report, include the bug reference, e.g.^^J%
 "latex/1234: ", preceding the subject text.^^J%
^^J%
  Your message will be entered into a publicly readable database^^J%
  Accessable via the www (see bugs.txt for details).^^J%
  If do not wish this message made public, Edit the^^J%
  >Confidential: no^^J%
  field to  yes  before submitting this message.^^J%
^^J%
  Thank you for taking the time to submit a bug report.}

\wmsg*{^^J%
============================================================^^J
^^J%
   End of LaTeX2e bug report.^^J%
============================================================}

%%
%% Close the .msg output stream.
%%
\immediate\closeout\msg

%%
%% This is the TeX primitive \end command.
%%
\@@end
" or some
%% similar command sequence rather than 
%%     latex latexbug
%% then the debugging info in \reserved@a will already have been lost.
%% This might not matter, but if it does we may ask the user to resubmit
%% the report.
\ifx\reserved@b\@undefined
  \ifx\reserved@a\@gobble
    \def\@inputfiles{NONE}
  \else
    \let\@inputfiles\reserved@a
  \fi
\else
  \def\@inputfiles{LOST}
\fi

%%
%% Output stream to produce the bug report template.
%%
\newwrite\msg
\immediate\openout\msg=\jobname.msg


%%
%% Check that LaTeX2e is being used.
%%
\ifx\undefined\newcommand
 \newlinechar`\^^J%
 \immediate\write17{^^J%
    You must use LaTeX2e to generate the bug report!^^J^^J%
    If there is a bug in the installation procedure,^^J%
    and you can not create LaTeX2e, you may use initex^^J%
    to generate the report}%

 \let\relax\end
\else
\def\@tempa{LaTeX2e}\ifx\@tempa\fmtname\else
 \immediate\write17{^^J%
  Older Versions of LaTeX are no longer supported.^^J%
  You must use LaTeX2e to generate the bug report!^^J^^J%
  If there is a bug in the installation procedure,^^J%
  and you can not create LaTeX2e, you may use initex^^J%
  to generate the report}%
 \let\relax\@@end
\fi\fi

%%
%% \wmsg writes to the terminal, and the .msg file
%% \wmsg* just writes to the .msg file
%% \typeout just writes to the terminal
%%

\def\wmsg{\bgroup\@ifstar{\interactivefalse\@wmsg}\@wmsg}

\fi

\relax
\endlinechar=-1

\def\@wmsg#1{%
  \ifinteractive\immediate\write17{#1}\fi
  \immediate\write\msg{#1}%
  \egroup}

%%
%% if \interactivefalse just make a blank template.
%%
\newif\ifinteractive
\interactivetrue

%%
%% Prompt for an answer from the user, if the answer is not
%% provided by the cfg file.
%%

\def\readifnotknown#1{%
 \@ifundefined{#1}%
    {{\message{#1> }%
     \catcode`\^^I=12 \let\do\@makeother\dospecials
     \global\read\m@ne t\expandafter o\csname#1\endcsname}}%
    {\message{\csname#1\endcsname}}}

%%
%% Pause so messages do not scroll off screen.
%%
\def\pause{%
  \ifinteractive
    \message{Press <return> key to continue. }%
    \read\m@ne to \@tempa
  \fi}

%%
%% Opening Banner.
%%
\typeout{^^J%
============================================================^^J%
^^J%
LaTeX bug report generator^^J%
==========================^^J%
Processing this file with LaTeX will produce a template \jobname.msg^^J
for submitting bug reports for the LaTeX distribution.^^J^^J
* Please do not report bugs in contributed, non-standard, files to the
  ^^J \space latex-bugs address.^^J
* Please write your report in English.^^J
* Please keep the report as short as possible.^^J%
* If possible, check whether the bug has already been reported.^^J %
  \space The bugs database is available on WWW:^^J \space
   http://www.latex-project.org/bugs.html^^J}


\ifinteractive
  \InputIfFileExists{latexbug.cfg}{\typeout{** latexbug.cfg used **}}{}
\fi

%% \batch is a `private' macro used to get a batchmode
%% (actually \nonstopmode) run for use with latexbug.el
\ifx\batch\undefined

\def\getcategory{%
\count@=0
\ifinteractive

\typeout{%
* Please carefully select the category as different categories^^J
\space\space are supported by different people!^^J^^J%
0) LaTeX:\@spaces
         The `base' format, and standard classes only (base).^^J
1) tools:\@spaces
         Packages supported by the LaTeX3 project team (tools).^^J
2) graphics:\space 
         The color and graphics packages (graphics).^^J
3) mfnfss: \space\space
         --- Don't use, no longer supported! ---^^J
4) psnfss: \space\space
         Packages for using PostScript fonts LaTeX (psnfss).^^J
5) amslatex:\space
         AMS supported Classes and Packages (amsfonts and amslatex).^^J
6) babel:\@spaces
         Packages supporting many different languages (babel).^^J%
7) expl3:\@spaces
         Experimental packages for TeX programmers. (expl3)^^J%
8) cyrillic: Packages for using Cyrillic fonts (cyrillic).^^J}

\message{Please select a category 0--8:  }
\read\m@ne to \answer
\if!\answer!\let\answer\m@ne\fi
\count@=\answer\relax
\else
\typeout{As you are using INITEX, I will assume category `latex'}
\fi

\ifcase\count@
\def\category{latex}\or
\def\category{tools}\or
\def\category{graphics}\or
\errhelp{Retry with <return>, or quit with x.}
\def\badcategory{This category is no longer supported}
\errmessage{\badcategory}
\expandafter\getcategory
\or
\def\category{psnfss}\or
\def\category{amslatex}\or
\def\category{babel}\or
\def\category{expl3}\or
\def\category{cyrillic}%
\else
\errhelp{Retry with <return>, or quit with x.}
\def\badcategory{Only categories 0,...,8 are supported at this time}
\errmessage{\badcategory}
\expandafter\getcategory
\fi}
\getcategory


\typeout{^^J%
============================================================^^J%
^^J%
Please give a one line ( < 50 character ) description of the problem.%
^^J^^J%
If you are using email to report the problem,^^J%
please also use this text as the `Subject' line for the mail message:%
^^J \@spaces\@spaces\space
                 |<------------------------------------------------>|} 


\loop
\let\synopsis\relax
\readifnotknown{synopsis}
\ifx\synopsis\@empty
\repeat


\typeout{%
^^J%
\ifinteractive
This report generator may be used in one of two ways.^^J%
If you choose the interactive option, you will be prompted to answer^^J%
several questions. Otherwise a blank template will be created for^^J%
you to fill in using your editor.^^J%
\else
INITEX should only be used for reporting bugs with the LaTeX2e^^J%
installation procedure. If you have a working copy of LaTeX2e,^^J%
please use that to generate the report.
\fi}

\ifinteractive
\typeout{Interactive session (y/n) ? }
\readifnotknown{interactive}

\ifx\interactive\@empty
   \def\interactive{n}
\fi

%%
%% Allow anything begining with `y' or `Y' for yes.
%%
\edef\interactive{\uccode`\expandafter\@car\interactive\@nil}
\ifnum \interactive=`Y \else\interactivefalse\fi
\else
\def\interactive{`\N}
\fi

\else
\def\category{< CATEGORY >}
\def\synopsis{< SYNOPSIS >}
\batchmode
\interactivefalse
\def\interactive{`\N}
\fi

%%
%% Header in the msg file.
%%
\wmsg*{^^J%
 LaTeX2e bug report.^^J%
\ifnum \interactive=`Y Generated \else Template generated \fi
 by latexbug.tex on \number\year/\two@digits\month/\two@digits\day^^J%
^^J%
 Reports may be submitted by email to latex-bugs@latex-project.org^^J%
 Please use the subject line:^^J%
 Subject: \synopsis^^J%
^^J%
 To follow up an existing report, include the bug reference, e.g.^^J
 "latex/1234: ", preceding the subject text.^^J
 ============================================================^^J}



%%
%% Category of bug, obtained earlier but put out now, after the header.
%%
\wmsg{>Category: \category}

%%
%% synopsis of bug, obtained earlier but put out now, after the header.
%%
\wmsg{>Synopsis: \synopsis}


%%
%% >Confidential: Default to no unless this is overridden
%% in latexbug.cfg. If you want to send a one-off confidential
%% report, just edit the latexbug.msg file to say yes.
%%
\wmsg{>Confidential: \ifx\confidential\undefined
                         no
                       \else
                         \confidential
                       \fi}


%% Try to catch various formats that babel has used to add to the banner
%% over the years. Currently it uses \typeout, so put the first \typeout
%% in \format and the second in \hyphenation. Earlier releases used 
%% \immedite\write
\begingroup
 \global\let\format\@empty
 \gdef\hyphenation{standard}
 \def\immediate#1#{\xdef\hyphenation}
 \def\typeout#1{%
    \xdef\format{\format#1}\def\typeout##1{\xdef\hyphenation{##1}}}
 \the\everyjob
\endgroup

\wmsg{>Release: \format}

\ifinteractive
%%
%% if interactive, \wread reads a line (verbatim) and writes it to the
%% .msg file, until two consecutive blank lines are entered.
%%
  \def\wread{%
    \begingroup
    \catcode`\^^I=12 
    \let\do\@makeother\dospecials
    \wreadloop}

  \def\wreadloop{%
    \let\lastanswer\answer
    \message{=> }\read\m@ne to \answer
    \ifx\lastanswer\@empty
      \let\lastanswer\answer
    \fi
    \ifx\lastanswer\@empty
      \endgroup
    \else
      \immediate\write\msg{\answer}%
      \expandafter\wreadloop
    \fi}
\else
%%
%% If non-interactive, \wread just writes a blank line to the .msg file,
%% and \wmsg does not write to the terminal.
%%
  \def\wread{\wmsg{}}
\fi

%%
%% \copytomsg copies the contents of a file into the .msg file.
%% (at least it does it as well as TeX can, so there may be
%% transcription problems with 8-bit characters).
%%
%% It does a line count, and complains if the test file is
%% too large.

\chardef\inputfile=15

\newcount\linecount

\def\copytomsg#1{{%
   \def\do##1{\catcode`##1=11}%
   \dospecials
   \global\linecount\z@
   \openin\inputfile#1\relax
   \def\thefile{#1}%
   \@copytomsg
   \closein\inputfile}}

\def\@copytomsg{%
   \ifeof\inputfile
      \typeout{*** \thefile\space line count = \the\linecount}
   \else
      \global\advance\linecount\@ne
      \read\inputfile to \inputline
      \ifx\inputline\@empty
         \wmsg*{}
      \else
         \wmsg*{\inputline}
      \fi
      \expandafter\@copytomsg
   \fi}


%%
%% Test the age of the current format.
%%
\def\getage#1/#2/#3\@nil{%
  \count@\year
  \advance\count@-#1\relax
  \multiply\count@ by 12\relax
  \advance\count@\month
  \advance\count@-#2\relax}
%
\expandafter\getage\fmtversion\@nil
%%
%% \count@ should now be the age of the format in months.
%%
%% Take a generous definition of `year'.
\ifnum\count@>65
\def\oldformat{^^J%
   ! Your LaTeX installation is more than five years old.^^J%
   ! Please consider updating LaTeX before submitting this report.^^J%
   ! At least check a current LaTeX changes.txt file, to see if the^^J%
   ! bug has been fixed in the current release.^^J%
   !}
%%
%% Put the message in a macro to improve the look of the error mesage.
%%

\errhelp{If you still wish to complete the form, just type return.}
\errmessage{\oldformat}
\fi

%%
%% Test fonts not customised too much. (unless using initex)
%%
\ifx\loop\relax\else
\edef\fontdefaults{%
  \encodingdefault/\familydefault/\seriesdefault/\shapedefault}
\def\standardfontdefaults{OT1/cmr/m/n}
\ifx\fontdefaults\standardfontdefaults\else
\wmsg*{>Unformatted:}
\wmsg{Normal font: \fontdefaults}
\def\customisedfonts{^^J%
! This format has customised font defaults.^^J%
! Please try to re-create the error using a standard format^^J%
! before submitting this report}
\errhelp{If you still wish to complete the form, just type return.}
\errmessage{\customisedfonts}
\fi
\fi

%%
%% Now use \wmsg and \wread for each of the multi-line fields 
%% in the form.
%% One line fields use \readifnotknown, which will only prompt
%% if the field has not already been set in the configuration file.
%%
\ifinteractive
  \typeout{^^JYour name:}
  \readifnotknown{name}
\else
  \ifx\name\undefined
    \def\name{ < ENTER YOUR NAME > }
  \fi
\fi


\ifinteractive
  \typeout{^^JYour Address (preferably email):}
  \readifnotknown{address}
\else
  \ifx\address\undefined
    \def\address{ < ENTER YOUR EMAIL ADDRESS > }
  \fi
\fi

\wmsg*{>Originator: \address \space(\name)}

%%
%% >Organisation: is really a GNATS multiline field
%% but we treat it as a one-line field.
%%
\wmsg*{>Organization: \ifx\organisation\undefined
                        \ifx\organization\undefined\else
                           \organization
                        \fi
                       \else
                         \organisation
                       \fi}


%%
%% Test which format is being used. These fields are completed
%% automatically even if the blank template is being produced.
%%

\wmsg*{>Environment:}
\wmsg*{ Hyphenation: \hyphenation}
\wmsg*{ \string\@TeXversion: \meaning\@TeXversion
        \ifx\@TeXversion\@@undefined
         \space (Standard setting for TeX3.141 and later)\fi}
\wmsg*{ \string\@currdir: \meaning\@currdir}
\wmsg*{ \string\input@path: \meaning\input@path
        \ifx\input@path\@@undefined
         \space (Standard setting)\fi}

\wmsg*{>Unformatted:}
\wmsg*{ *** Initex configuration files}
%%
%% These are in a comma separated list, so locally reset
%% \newlinechar
{\newlinechar`\,
 \wmsg*{\@inputfiles}}
\wmsg*{ ***}

\wmsg*{>Description:}
\typeout{}
\wmsg{Description of bug:}
\ifinteractive
  \typeout{%
    \@spaces The answer to this  question may take several lines.^^J%
    \@spaces (Each such line will be prompted by =>.)^^J%
    \@spaces Typing TWO consecutive blank lines terminates the answer.}
\else
\wmsg{ < ENTER BUG REPORT HERE >}
\fi
\wread



%%
%% insertion of the test file
%%



\ifinteractive
   \typeout{^^J%
 Name of a SHORT, SELF-CONTAINED file which indicates the problem:^^J%
 This file should be as small as possible (preferably < 60 lines)^^J%
 Any non-standard files that the test file uses should be included^^J%
 using the filecontents environment.^^J^^J%
%
 LaTeX will try to input this file, so give the full path^^J%
 if the file is not in the current directory.^^J^^J%
%
 If you are not reporting a bug, and there is therefore^^J%
 no test file, just hit <return>}
   \message{filename> }\read\m@ne to \filename
\else
   \def\filename{}
\fi

%% 
%% Try to find the .tex file and .log file
%%


\ifx\filename\@empty
  \ifx\LaTeX\undefinedcommand
  \else
    \ifinteractive
      \typeout{^^J^^JNo test file.^^J^^J%
      Three classes of report are supported:^^J^^J%
      0) sw-bug:^^J\@spaces
         Bug in the software, the report should include a test file.^^J
      1) doc-bug:^^J\@spaces
         Inaccuracies in the documentation.^^J
      2) change-request:^^J\@spaces 
         Not a bug, but rather a request for LaTeX to be changed.^^J}
      \message{Please select a category 0--2:  }
      \read\m@ne to \answer
      \ifx\answer\@empty
        \def\answer{-1}
      \fi
      \count@=\answer\relax
      \else
        \count@=\z@
      \fi
    \ifcase\count@
      \ifinteractive\wmsg{>Class: sw-bug}\fi
      \typeout{^^J! Please edit the message to add a test file and log!}
      \pause
      \wmsg*{^^J>How-To-Repeat:}
      \wmsg*{%
      Sample file which indicates the problem:^^J%
      ========================================^^J%
      \space< TEST FILE HERE >^^J%
      ^^J%
      The log file from running LaTeX on the sample:^^J%
      ==============================================^^J%
      \space< LOG FROM TEST FILE HERE >}
    \or
      \wmsg{>Class: doc-bug}
    \or
      \wmsg{>Class: change-request}
    \else
      \errhelp{Quit with `x' and then re-start latexbug}
      \def\badcategory{Only classes 0,1,2 are supported at this time}
      \errmessage{\badcategory}
    \fi
  \fi
\else

\filename@parse\filename

\IfFileExists{\filename}{\edef\samplefile{\filename}}{}

\IfFileExists{\filename@area\filename@base.log}
  {\edef\logfile{\filename@area\filename@base.log}}
  {\IfFileExists{\filename@area\filename@base.lis}
    {\edef\logfile{\filename@area\filename@base.lis}}
    {}}


%% 
%% The example file goes here:
%%
\wmsg*{^^J>How-To-Repeat:}

\wmsg*{^^J%
Sample file which indicates the problem:^^J%
========================================}

\ifx\samplefile\undefinedcommand
   \typeout{^^J%
      Sample file \filename\space not found.^^J%
      Please edit \jobname.msg to include the sample file.}
   \wmsg*{ < TEST FILE HERE >}
   \pause
\else
   \copytomsg{\samplefile}
   \ifnum\linecount>60
    \typeout{%
^^J%
!!! Your test file is \the\linecount\space lines long.^^J%
!!! Such a large test file causes us problems:^^J%
!!! * It makes it difficult to track down the error^^J%
!!! * It makes our database for storing reports unnecessarily large.^^J%
!!! ^^J%
!!! Please, if at all possible, cut down your test file to the^^J%
!!! smallest file that shows the behaviour.^^J}
   \pause
   \fi
\fi


%%
%% The log file goes here:
%%
\wmsg*{^^J%
The log file from running LaTeX on the sample:^^J%
==============================================}

\ifx\logfile\undefinedcommand
   \typeout{^^J%
      Log file \filename@area\filename@base.log not found.^^J%
      Please edit \jobname.msg to include the log file.}
   \wmsg*{ < < LOG FROM TEST FILE HERE >}
   \pause
\else
   \copytomsg{\logfile}
\fi

\fi


%%
%% Closing Banner.
%%
\typeout{^^J%
============================================================}

\ifinteractive
 \typeout{^^J%
   You may wish to make further changes to the bug report file:^^J%
   `\jobname.msg'^^J%
   using your editor.}
\else
 \typeout{^^J%
   A template for submitting bug reports has been left in the file:^^J%
   \jobname.msg^^J%
   Please use your editor to complete the file before submitting^^J%
   your report.}
\fi

\let\ifinteractivetrue\iftrue
\typeout{^^J%
  If you have access to email, please send `\jobname.msg' to:^^J%
  latex-bugs@latex-project.org  Please use the subject line:^^J%
  \@spaces Subject: \synopsis^^J%
^^J%
 (This subject will be used in all subsequent correspondence.)^^J%
^^J%
 To follow up an existing report, include the bug reference, e.g.^^J%
 "latex/1234: ", preceding the subject text.^^J%
^^J%
  Your message will be entered into a publicly readable database^^J%
  Accessable via the www (see bugs.txt for details).^^J%
  If do not wish this message made public, Edit the^^J%
  >Confidential: no^^J%
  field to  yes  before submitting this message.^^J%
^^J%
  Thank you for taking the time to submit a bug report.}

\wmsg*{^^J%
============================================================^^J
^^J%
   End of LaTeX2e bug report.^^J%
============================================================}

%%
%% Close the .msg output stream.
%%
\immediate\closeout\msg

%%
%% This is the TeX primitive \end command.
%%
\@@end
" or some
%% similar command sequence rather than 
%%     latex latexbug
%% then the debugging info in \reserved@a will already have been lost.
%% This might not matter, but if it does we may ask the user to resubmit
%% the report.
\ifx\reserved@b\@undefined
  \ifx\reserved@a\@gobble
    \def\@inputfiles{NONE}
  \else
    \let\@inputfiles\reserved@a
  \fi
\else
  \def\@inputfiles{LOST}
\fi

%%
%% Output stream to produce the bug report template.
%%
\newwrite\msg
\immediate\openout\msg=\jobname.msg


%%
%% Check that LaTeX2e is being used.
%%
\ifx\undefined\newcommand
 \newlinechar`\^^J%
 \immediate\write17{^^J%
    You must use LaTeX2e to generate the bug report!^^J^^J%
    If there is a bug in the installation procedure,^^J%
    and you can not create LaTeX2e, you may use initex^^J%
    to generate the report}%

 \let\relax\end
\else
\def\@tempa{LaTeX2e}\ifx\@tempa\fmtname\else
 \immediate\write17{^^J%
  Older Versions of LaTeX are no longer supported.^^J%
  You must use LaTeX2e to generate the bug report!^^J^^J%
  If there is a bug in the installation procedure,^^J%
  and you can not create LaTeX2e, you may use initex^^J%
  to generate the report}%
 \let\relax\@@end
\fi\fi

%%
%% \wmsg writes to the terminal, and the .msg file
%% \wmsg* just writes to the .msg file
%% \typeout just writes to the terminal
%%

\def\wmsg{\bgroup\@ifstar{\interactivefalse\@wmsg}\@wmsg}

\fi

\relax
\endlinechar=-1

\def\@wmsg#1{%
  \ifinteractive\immediate\write17{#1}\fi
  \immediate\write\msg{#1}%
  \egroup}

%%
%% if \interactivefalse just make a blank template.
%%
\newif\ifinteractive
\interactivetrue

%%
%% Prompt for an answer from the user, if the answer is not
%% provided by the cfg file.
%%

\def\readifnotknown#1{%
 \@ifundefined{#1}%
    {{\message{#1> }%
     \catcode`\^^I=12 \let\do\@makeother\dospecials
     \global\read\m@ne t\expandafter o\csname#1\endcsname}}%
    {\message{\csname#1\endcsname}}}

%%
%% Pause so messages do not scroll off screen.
%%
\def\pause{%
  \ifinteractive
    \message{Press <return> key to continue. }%
    \read\m@ne to \@tempa
  \fi}

%%
%% Opening Banner.
%%
\typeout{^^J%
============================================================^^J%
^^J%
LaTeX bug report generator^^J%
==========================^^J%
Processing this file with LaTeX will produce a template \jobname.msg^^J
for submitting bug reports for the LaTeX distribution.^^J^^J
* Please do not report bugs in contributed, non-standard, files to the
  ^^J \space latex-bugs address.^^J
* Please write your report in English.^^J
* Please keep the report as short as possible.^^J%
* If possible, check whether the bug has already been reported.^^J %
  \space The bugs database is available on WWW:^^J \space
   http://www.latex-project.org/bugs.html^^J}


\ifinteractive
  \InputIfFileExists{latexbug.cfg}{\typeout{** latexbug.cfg used **}}{}
\fi

%% \batch is a `private' macro used to get a batchmode
%% (actually \nonstopmode) run for use with latexbug.el
\ifx\batch\undefined

\def\getcategory{%
\count@=0
\ifinteractive

\typeout{%
* Please carefully select the category as different categories^^J
\space\space are supported by different people!^^J^^J%
0) LaTeX:\@spaces
         The `base' format, and standard classes only (base).^^J
1) tools:\@spaces
         Packages supported by the LaTeX3 project team (tools).^^J
2) graphics:\space 
         The color and graphics packages (graphics).^^J
3) mfnfss: \space\space
         --- Don't use, no longer supported! ---^^J
4) psnfss: \space\space
         Packages for using PostScript fonts LaTeX (psnfss).^^J
5) amslatex:\space
         AMS supported Classes and Packages (amsfonts and amslatex).^^J
6) babel:\@spaces
         Packages supporting many different languages (babel).^^J%
7) expl3:\@spaces
         Experimental packages for TeX programmers. (expl3)^^J%
8) cyrillic: Packages for using Cyrillic fonts (cyrillic).^^J}

\message{Please select a category 0--8:  }
\read\m@ne to \answer
\if!\answer!\let\answer\m@ne\fi
\count@=\answer\relax
\else
\typeout{As you are using INITEX, I will assume category `latex'}
\fi

\ifcase\count@
\def\category{latex}\or
\def\category{tools}\or
\def\category{graphics}\or
\errhelp{Retry with <return>, or quit with x.}
\def\badcategory{This category is no longer supported}
\errmessage{\badcategory}
\expandafter\getcategory
\or
\def\category{psnfss}\or
\def\category{amslatex}\or
\def\category{babel}\or
\def\category{expl3}\or
\def\category{cyrillic}%
\else
\errhelp{Retry with <return>, or quit with x.}
\def\badcategory{Only categories 0,...,8 are supported at this time}
\errmessage{\badcategory}
\expandafter\getcategory
\fi}
\getcategory


\typeout{^^J%
============================================================^^J%
^^J%
Please give a one line ( < 50 character ) description of the problem.%
^^J^^J%
If you are using email to report the problem,^^J%
please also use this text as the `Subject' line for the mail message:%
^^J \@spaces\@spaces\space
                 |<------------------------------------------------>|} 


\loop
\let\synopsis\relax
\readifnotknown{synopsis}
\ifx\synopsis\@empty
\repeat


\typeout{%
^^J%
\ifinteractive
This report generator may be used in one of two ways.^^J%
If you choose the interactive option, you will be prompted to answer^^J%
several questions. Otherwise a blank template will be created for^^J%
you to fill in using your editor.^^J%
\else
INITEX should only be used for reporting bugs with the LaTeX2e^^J%
installation procedure. If you have a working copy of LaTeX2e,^^J%
please use that to generate the report.
\fi}

\ifinteractive
\typeout{Interactive session (y/n) ? }
\readifnotknown{interactive}

\ifx\interactive\@empty
   \def\interactive{n}
\fi

%%
%% Allow anything begining with `y' or `Y' for yes.
%%
\edef\interactive{\uccode`\expandafter\@car\interactive\@nil}
\ifnum \interactive=`Y \else\interactivefalse\fi
\else
\def\interactive{`\N}
\fi

\else
\def\category{< CATEGORY >}
\def\synopsis{< SYNOPSIS >}
\batchmode
\interactivefalse
\def\interactive{`\N}
\fi

%%
%% Header in the msg file.
%%
\wmsg*{^^J%
 LaTeX2e bug report.^^J%
\ifnum \interactive=`Y Generated \else Template generated \fi
 by latexbug.tex on \number\year/\two@digits\month/\two@digits\day^^J%
^^J%
 Reports may be submitted by email to latex-bugs@latex-project.org^^J%
 Please use the subject line:^^J%
 Subject: \synopsis^^J%
^^J%
 To follow up an existing report, include the bug reference, e.g.^^J
 "latex/1234: ", preceding the subject text.^^J
 ============================================================^^J}



%%
%% Category of bug, obtained earlier but put out now, after the header.
%%
\wmsg{>Category: \category}

%%
%% synopsis of bug, obtained earlier but put out now, after the header.
%%
\wmsg{>Synopsis: \synopsis}


%%
%% >Confidential: Default to no unless this is overridden
%% in latexbug.cfg. If you want to send a one-off confidential
%% report, just edit the latexbug.msg file to say yes.
%%
\wmsg{>Confidential: \ifx\confidential\undefined
                         no
                       \else
                         \confidential
                       \fi}


%% Try to catch various formats that babel has used to add to the banner
%% over the years. Currently it uses \typeout, so put the first \typeout
%% in \format and the second in \hyphenation. Earlier releases used 
%% \immedite\write
\begingroup
 \global\let\format\@empty
 \gdef\hyphenation{standard}
 \def\immediate#1#{\xdef\hyphenation}
 \def\typeout#1{%
    \xdef\format{\format#1}\def\typeout##1{\xdef\hyphenation{##1}}}
 \the\everyjob
\endgroup

\wmsg{>Release: \format}

\ifinteractive
%%
%% if interactive, \wread reads a line (verbatim) and writes it to the
%% .msg file, until two consecutive blank lines are entered.
%%
  \def\wread{%
    \begingroup
    \catcode`\^^I=12 
    \let\do\@makeother\dospecials
    \wreadloop}

  \def\wreadloop{%
    \let\lastanswer\answer
    \message{=> }\read\m@ne to \answer
    \ifx\lastanswer\@empty
      \let\lastanswer\answer
    \fi
    \ifx\lastanswer\@empty
      \endgroup
    \else
      \immediate\write\msg{\answer}%
      \expandafter\wreadloop
    \fi}
\else
%%
%% If non-interactive, \wread just writes a blank line to the .msg file,
%% and \wmsg does not write to the terminal.
%%
  \def\wread{\wmsg{}}
\fi

%%
%% \copytomsg copies the contents of a file into the .msg file.
%% (at least it does it as well as TeX can, so there may be
%% transcription problems with 8-bit characters).
%%
%% It does a line count, and complains if the test file is
%% too large.

\chardef\inputfile=15

\newcount\linecount

\def\copytomsg#1{{%
   \def\do##1{\catcode`##1=11}%
   \dospecials
   \global\linecount\z@
   \openin\inputfile#1\relax
   \def\thefile{#1}%
   \@copytomsg
   \closein\inputfile}}

\def\@copytomsg{%
   \ifeof\inputfile
      \typeout{*** \thefile\space line count = \the\linecount}
   \else
      \global\advance\linecount\@ne
      \read\inputfile to \inputline
      \ifx\inputline\@empty
         \wmsg*{}
      \else
         \wmsg*{\inputline}
      \fi
      \expandafter\@copytomsg
   \fi}


%%
%% Test the age of the current format.
%%
\def\getage#1/#2/#3\@nil{%
  \count@\year
  \advance\count@-#1\relax
  \multiply\count@ by 12\relax
  \advance\count@\month
  \advance\count@-#2\relax}
%
\expandafter\getage\fmtversion\@nil
%%
%% \count@ should now be the age of the format in months.
%%
%% Take a generous definition of `year'.
\ifnum\count@>65
\def\oldformat{^^J%
   ! Your LaTeX installation is more than five years old.^^J%
   ! Please consider updating LaTeX before submitting this report.^^J%
   ! At least check a current LaTeX changes.txt file, to see if the^^J%
   ! bug has been fixed in the current release.^^J%
   !}
%%
%% Put the message in a macro to improve the look of the error mesage.
%%

\errhelp{If you still wish to complete the form, just type return.}
\errmessage{\oldformat}
\fi

%%
%% Test fonts not customised too much. (unless using initex)
%%
\ifx\loop\relax\else
\edef\fontdefaults{%
  \encodingdefault/\familydefault/\seriesdefault/\shapedefault}
\def\standardfontdefaults{OT1/cmr/m/n}
\ifx\fontdefaults\standardfontdefaults\else
\wmsg*{>Unformatted:}
\wmsg{Normal font: \fontdefaults}
\def\customisedfonts{^^J%
! This format has customised font defaults.^^J%
! Please try to re-create the error using a standard format^^J%
! before submitting this report}
\errhelp{If you still wish to complete the form, just type return.}
\errmessage{\customisedfonts}
\fi
\fi

%%
%% Now use \wmsg and \wread for each of the multi-line fields 
%% in the form.
%% One line fields use \readifnotknown, which will only prompt
%% if the field has not already been set in the configuration file.
%%
\ifinteractive
  \typeout{^^JYour name:}
  \readifnotknown{name}
\else
  \ifx\name\undefined
    \def\name{ < ENTER YOUR NAME > }
  \fi
\fi


\ifinteractive
  \typeout{^^JYour Address (preferably email):}
  \readifnotknown{address}
\else
  \ifx\address\undefined
    \def\address{ < ENTER YOUR EMAIL ADDRESS > }
  \fi
\fi

\wmsg*{>Originator: \address \space(\name)}

%%
%% >Organisation: is really a GNATS multiline field
%% but we treat it as a one-line field.
%%
\wmsg*{>Organization: \ifx\organisation\undefined
                        \ifx\organization\undefined\else
                           \organization
                        \fi
                       \else
                         \organisation
                       \fi}


%%
%% Test which format is being used. These fields are completed
%% automatically even if the blank template is being produced.
%%

\wmsg*{>Environment:}
\wmsg*{ Hyphenation: \hyphenation}
\wmsg*{ \string\@TeXversion: \meaning\@TeXversion
        \ifx\@TeXversion\@@undefined
         \space (Standard setting for TeX3.141 and later)\fi}
\wmsg*{ \string\@currdir: \meaning\@currdir}
\wmsg*{ \string\input@path: \meaning\input@path
        \ifx\input@path\@@undefined
         \space (Standard setting)\fi}

\wmsg*{>Unformatted:}
\wmsg*{ *** Initex configuration files}
%%
%% These are in a comma separated list, so locally reset
%% \newlinechar
{\newlinechar`\,
 \wmsg*{\@inputfiles}}
\wmsg*{ ***}

\wmsg*{>Description:}
\typeout{}
\wmsg{Description of bug:}
\ifinteractive
  \typeout{%
    \@spaces The answer to this  question may take several lines.^^J%
    \@spaces (Each such line will be prompted by =>.)^^J%
    \@spaces Typing TWO consecutive blank lines terminates the answer.}
\else
\wmsg{ < ENTER BUG REPORT HERE >}
\fi
\wread



%%
%% insertion of the test file
%%



\ifinteractive
   \typeout{^^J%
 Name of a SHORT, SELF-CONTAINED file which indicates the problem:^^J%
 This file should be as small as possible (preferably < 60 lines)^^J%
 Any non-standard files that the test file uses should be included^^J%
 using the filecontents environment.^^J^^J%
%
 LaTeX will try to input this file, so give the full path^^J%
 if the file is not in the current directory.^^J^^J%
%
 If you are not reporting a bug, and there is therefore^^J%
 no test file, just hit <return>}
   \message{filename> }\read\m@ne to \filename
\else
   \def\filename{}
\fi

%% 
%% Try to find the .tex file and .log file
%%


\ifx\filename\@empty
  \ifx\LaTeX\undefinedcommand
  \else
    \ifinteractive
      \typeout{^^J^^JNo test file.^^J^^J%
      Three classes of report are supported:^^J^^J%
      0) sw-bug:^^J\@spaces
         Bug in the software, the report should include a test file.^^J
      1) doc-bug:^^J\@spaces
         Inaccuracies in the documentation.^^J
      2) change-request:^^J\@spaces 
         Not a bug, but rather a request for LaTeX to be changed.^^J}
      \message{Please select a category 0--2:  }
      \read\m@ne to \answer
      \ifx\answer\@empty
        \def\answer{-1}
      \fi
      \count@=\answer\relax
      \else
        \count@=\z@
      \fi
    \ifcase\count@
      \ifinteractive\wmsg{>Class: sw-bug}\fi
      \typeout{^^J! Please edit the message to add a test file and log!}
      \pause
      \wmsg*{^^J>How-To-Repeat:}
      \wmsg*{%
      Sample file which indicates the problem:^^J%
      ========================================^^J%
      \space< TEST FILE HERE >^^J%
      ^^J%
      The log file from running LaTeX on the sample:^^J%
      ==============================================^^J%
      \space< LOG FROM TEST FILE HERE >}
    \or
      \wmsg{>Class: doc-bug}
    \or
      \wmsg{>Class: change-request}
    \else
      \errhelp{Quit with `x' and then re-start latexbug}
      \def\badcategory{Only classes 0,1,2 are supported at this time}
      \errmessage{\badcategory}
    \fi
  \fi
\else

\filename@parse\filename

\IfFileExists{\filename}{\edef\samplefile{\filename}}{}

\IfFileExists{\filename@area\filename@base.log}
  {\edef\logfile{\filename@area\filename@base.log}}
  {\IfFileExists{\filename@area\filename@base.lis}
    {\edef\logfile{\filename@area\filename@base.lis}}
    {}}


%% 
%% The example file goes here:
%%
\wmsg*{^^J>How-To-Repeat:}

\wmsg*{^^J%
Sample file which indicates the problem:^^J%
========================================}

\ifx\samplefile\undefinedcommand
   \typeout{^^J%
      Sample file \filename\space not found.^^J%
      Please edit \jobname.msg to include the sample file.}
   \wmsg*{ < TEST FILE HERE >}
   \pause
\else
   \copytomsg{\samplefile}
   \ifnum\linecount>60
    \typeout{%
^^J%
!!! Your test file is \the\linecount\space lines long.^^J%
!!! Such a large test file causes us problems:^^J%
!!! * It makes it difficult to track down the error^^J%
!!! * It makes our database for storing reports unnecessarily large.^^J%
!!! ^^J%
!!! Please, if at all possible, cut down your test file to the^^J%
!!! smallest file that shows the behaviour.^^J}
   \pause
   \fi
\fi


%%
%% The log file goes here:
%%
\wmsg*{^^J%
The log file from running LaTeX on the sample:^^J%
==============================================}

\ifx\logfile\undefinedcommand
   \typeout{^^J%
      Log file \filename@area\filename@base.log not found.^^J%
      Please edit \jobname.msg to include the log file.}
   \wmsg*{ < < LOG FROM TEST FILE HERE >}
   \pause
\else
   \copytomsg{\logfile}
\fi

\fi


%%
%% Closing Banner.
%%
\typeout{^^J%
============================================================}

\ifinteractive
 \typeout{^^J%
   You may wish to make further changes to the bug report file:^^J%
   `\jobname.msg'^^J%
   using your editor.}
\else
 \typeout{^^J%
   A template for submitting bug reports has been left in the file:^^J%
   \jobname.msg^^J%
   Please use your editor to complete the file before submitting^^J%
   your report.}
\fi

\let\ifinteractivetrue\iftrue
\typeout{^^J%
  If you have access to email, please send `\jobname.msg' to:^^J%
  latex-bugs@latex-project.org  Please use the subject line:^^J%
  \@spaces Subject: \synopsis^^J%
^^J%
 (This subject will be used in all subsequent correspondence.)^^J%
^^J%
 To follow up an existing report, include the bug reference, e.g.^^J%
 "latex/1234: ", preceding the subject text.^^J%
^^J%
  Your message will be entered into a publicly readable database^^J%
  Accessable via the www (see bugs.txt for details).^^J%
  If do not wish this message made public, Edit the^^J%
  >Confidential: no^^J%
  field to  yes  before submitting this message.^^J%
^^J%
  Thank you for taking the time to submit a bug report.}

\wmsg*{^^J%
============================================================^^J
^^J%
   End of LaTeX2e bug report.^^J%
============================================================}

%%
%% Close the .msg output stream.
%%
\immediate\closeout\msg

%%
%% This is the TeX primitive \end command.
%%
\@@end
" or some
%% similar command sequence rather than 
%%     latex latexbug
%% then the debugging info in \reserved@a will already have been lost.
%% This might not matter, but if it does we may ask the user to resubmit
%% the report.
\ifx\reserved@b\@undefined
  \ifx\reserved@a\@gobble
    \def\@inputfiles{NONE}
  \else
    \let\@inputfiles\reserved@a
  \fi
\else
  \def\@inputfiles{LOST}
\fi

%%
%% Output stream to produce the bug report template.
%%
\newwrite\msg
\immediate\openout\msg=\jobname.msg


%%
%% Check that LaTeX2e is being used.
%%
\ifx\undefined\newcommand
 \newlinechar`\^^J%
 \immediate\write17{^^J%
    You must use LaTeX2e to generate the bug report!^^J^^J%
    If there is a bug in the installation procedure,^^J%
    and you can not create LaTeX2e, you may use initex^^J%
    to generate the report}%

 \let\relax\end
\else
\def\@tempa{LaTeX2e}\ifx\@tempa\fmtname\else
 \immediate\write17{^^J%
  Older Versions of LaTeX are no longer supported.^^J%
  You must use LaTeX2e to generate the bug report!^^J^^J%
  If there is a bug in the installation procedure,^^J%
  and you can not create LaTeX2e, you may use initex^^J%
  to generate the report}%
 \let\relax\@@end
\fi\fi

%%
%% \wmsg writes to the terminal, and the .msg file
%% \wmsg* just writes to the .msg file
%% \typeout just writes to the terminal
%%

\def\wmsg{\bgroup\@ifstar{\interactivefalse\@wmsg}\@wmsg}

\fi

\relax
\endlinechar=-1

\def\@wmsg#1{%
  \ifinteractive\immediate\write17{#1}\fi
  \immediate\write\msg{#1}%
  \egroup}

%%
%% if \interactivefalse just make a blank template.
%%
\newif\ifinteractive
\interactivetrue

%%
%% Prompt for an answer from the user, if the answer is not
%% provided by the cfg file.
%%

\def\readifnotknown#1{%
 \@ifundefined{#1}%
    {{\message{#1> }%
     \catcode`\^^I=12 \let\do\@makeother\dospecials
     \global\read\m@ne t\expandafter o\csname#1\endcsname}}%
    {\message{\csname#1\endcsname}}}

%%
%% Pause so messages do not scroll off screen.
%%
\def\pause{%
  \ifinteractive
    \message{Press <return> key to continue. }%
    \read\m@ne to \@tempa
  \fi}

%%
%% Opening Banner.
%%
\typeout{^^J%
============================================================^^J%
^^J%
LaTeX bug report generator^^J%
==========================^^J%
Processing this file with LaTeX will produce a template \jobname.msg^^J
for submitting bug reports for the LaTeX distribution.^^J^^J
* Please do not report bugs in contributed, non-standard, files to the
  ^^J \space latex-bugs address.^^J
* Please write your report in English.^^J
* Please keep the report as short as possible.^^J%
* If possible, check whether the bug has already been reported.^^J %
  \space The bugs database is available on WWW:^^J \space
   http://www.latex-project.org/bugs.html^^J}


\ifinteractive
  \InputIfFileExists{latexbug.cfg}{\typeout{** latexbug.cfg used **}}{}
\fi

%% \batch is a `private' macro used to get a batchmode
%% (actually \nonstopmode) run for use with latexbug.el
\ifx\batch\undefined

\def\getcategory{%
\count@=0
\ifinteractive

\typeout{%
* Please carefully select the category as different categories^^J
\space\space are supported by different people!^^J^^J%
0) LaTeX:\@spaces
         The `base' format, and standard classes only (base).^^J
1) tools:\@spaces
         Packages supported by the LaTeX3 project team (tools).^^J
2) graphics:\space 
         The color and graphics packages (graphics).^^J
3) mfnfss: \space\space
         --- Don't use, no longer supported! ---^^J
4) psnfss: \space\space
         Packages for using PostScript fonts LaTeX (psnfss).^^J
5) amslatex:\space
         AMS supported Classes and Packages (amsfonts and amslatex).^^J
6) babel:\@spaces
         Packages supporting many different languages (babel).^^J%
7) expl3:\@spaces
         Experimental packages for TeX programmers. (expl3)^^J%
8) cyrillic: Packages for using Cyrillic fonts (cyrillic).^^J}

\message{Please select a category 0--8:  }
\read\m@ne to \answer
\if!\answer!\let\answer\m@ne\fi
\count@=\answer\relax
\else
\typeout{As you are using INITEX, I will assume category `latex'}
\fi

\ifcase\count@
\def\category{latex}\or
\def\category{tools}\or
\def\category{graphics}\or
\errhelp{Retry with <return>, or quit with x.}
\def\badcategory{This category is no longer supported}
\errmessage{\badcategory}
\expandafter\getcategory
\or
\def\category{psnfss}\or
\def\category{amslatex}\or
\def\category{babel}\or
\def\category{expl3}\or
\def\category{cyrillic}%
\else
\errhelp{Retry with <return>, or quit with x.}
\def\badcategory{Only categories 0,...,8 are supported at this time}
\errmessage{\badcategory}
\expandafter\getcategory
\fi}
\getcategory


\typeout{^^J%
============================================================^^J%
^^J%
Please give a one line ( < 50 character ) description of the problem.%
^^J^^J%
If you are using email to report the problem,^^J%
please also use this text as the `Subject' line for the mail message:%
^^J \@spaces\@spaces\space
                 |<------------------------------------------------>|} 


\loop
\let\synopsis\relax
\readifnotknown{synopsis}
\ifx\synopsis\@empty
\repeat


\typeout{%
^^J%
\ifinteractive
This report generator may be used in one of two ways.^^J%
If you choose the interactive option, you will be prompted to answer^^J%
several questions. Otherwise a blank template will be created for^^J%
you to fill in using your editor.^^J%
\else
INITEX should only be used for reporting bugs with the LaTeX2e^^J%
installation procedure. If you have a working copy of LaTeX2e,^^J%
please use that to generate the report.
\fi}

\ifinteractive
\typeout{Interactive session (y/n) ? }
\readifnotknown{interactive}

\ifx\interactive\@empty
   \def\interactive{n}
\fi

%%
%% Allow anything begining with `y' or `Y' for yes.
%%
\edef\interactive{\uccode`\expandafter\@car\interactive\@nil}
\ifnum \interactive=`Y \else\interactivefalse\fi
\else
\def\interactive{`\N}
\fi

\else
\def\category{< CATEGORY >}
\def\synopsis{< SYNOPSIS >}
\batchmode
\interactivefalse
\def\interactive{`\N}
\fi

%%
%% Header in the msg file.
%%
\wmsg*{^^J%
 LaTeX2e bug report.^^J%
\ifnum \interactive=`Y Generated \else Template generated \fi
 by latexbug.tex on \number\year/\two@digits\month/\two@digits\day^^J%
^^J%
 Reports may be submitted by email to latex-bugs@latex-project.org^^J%
 Please use the subject line:^^J%
 Subject: \synopsis^^J%
^^J%
 To follow up an existing report, include the bug reference, e.g.^^J
 "latex/1234: ", preceding the subject text.^^J
 ============================================================^^J}



%%
%% Category of bug, obtained earlier but put out now, after the header.
%%
\wmsg{>Category: \category}

%%
%% synopsis of bug, obtained earlier but put out now, after the header.
%%
\wmsg{>Synopsis: \synopsis}


%%
%% >Confidential: Default to no unless this is overridden
%% in latexbug.cfg. If you want to send a one-off confidential
%% report, just edit the latexbug.msg file to say yes.
%%
\wmsg{>Confidential: \ifx\confidential\undefined
                         no
                       \else
                         \confidential
                       \fi}


%% Try to catch various formats that babel has used to add to the banner
%% over the years. Currently it uses \typeout, so put the first \typeout
%% in \format and the second in \hyphenation. Earlier releases used 
%% \immedite\write
\begingroup
 \global\let\format\@empty
 \gdef\hyphenation{standard}
 \def\immediate#1#{\xdef\hyphenation}
 \def\typeout#1{%
    \xdef\format{\format#1}\def\typeout##1{\xdef\hyphenation{##1}}}
 \the\everyjob
\endgroup

\wmsg{>Release: \format}

\ifinteractive
%%
%% if interactive, \wread reads a line (verbatim) and writes it to the
%% .msg file, until two consecutive blank lines are entered.
%%
  \def\wread{%
    \begingroup
    \catcode`\^^I=12 
    \let\do\@makeother\dospecials
    \wreadloop}

  \def\wreadloop{%
    \let\lastanswer\answer
    \message{=> }\read\m@ne to \answer
    \ifx\lastanswer\@empty
      \let\lastanswer\answer
    \fi
    \ifx\lastanswer\@empty
      \endgroup
    \else
      \immediate\write\msg{\answer}%
      \expandafter\wreadloop
    \fi}
\else
%%
%% If non-interactive, \wread just writes a blank line to the .msg file,
%% and \wmsg does not write to the terminal.
%%
  \def\wread{\wmsg{}}
\fi

%%
%% \copytomsg copies the contents of a file into the .msg file.
%% (at least it does it as well as TeX can, so there may be
%% transcription problems with 8-bit characters).
%%
%% It does a line count, and complains if the test file is
%% too large.

\chardef\inputfile=15

\newcount\linecount

\def\copytomsg#1{{%
   \def\do##1{\catcode`##1=11}%
   \dospecials
   \global\linecount\z@
   \openin\inputfile#1\relax
   \def\thefile{#1}%
   \@copytomsg
   \closein\inputfile}}

\def\@copytomsg{%
   \ifeof\inputfile
      \typeout{*** \thefile\space line count = \the\linecount}
   \else
      \global\advance\linecount\@ne
      \read\inputfile to \inputline
      \ifx\inputline\@empty
         \wmsg*{}
      \else
         \wmsg*{\inputline}
      \fi
      \expandafter\@copytomsg
   \fi}


%%
%% Test the age of the current format.
%%
\def\getage#1/#2/#3\@nil{%
  \count@\year
  \advance\count@-#1\relax
  \multiply\count@ by 12\relax
  \advance\count@\month
  \advance\count@-#2\relax}
%
\expandafter\getage\fmtversion\@nil
%%
%% \count@ should now be the age of the format in months.
%%
%% Take a generous definition of `year'.
\ifnum\count@>65
\def\oldformat{^^J%
   ! Your LaTeX installation is more than five years old.^^J%
   ! Please consider updating LaTeX before submitting this report.^^J%
   ! At least check a current LaTeX changes.txt file, to see if the^^J%
   ! bug has been fixed in the current release.^^J%
   !}
%%
%% Put the message in a macro to improve the look of the error mesage.
%%

\errhelp{If you still wish to complete the form, just type return.}
\errmessage{\oldformat}
\fi

%%
%% Test fonts not customised too much. (unless using initex)
%%
\ifx\loop\relax\else
\edef\fontdefaults{%
  \encodingdefault/\familydefault/\seriesdefault/\shapedefault}
\def\standardfontdefaults{OT1/cmr/m/n}
\ifx\fontdefaults\standardfontdefaults\else
\wmsg*{>Unformatted:}
\wmsg{Normal font: \fontdefaults}
\def\customisedfonts{^^J%
! This format has customised font defaults.^^J%
! Please try to re-create the error using a standard format^^J%
! before submitting this report}
\errhelp{If you still wish to complete the form, just type return.}
\errmessage{\customisedfonts}
\fi
\fi

%%
%% Now use \wmsg and \wread for each of the multi-line fields 
%% in the form.
%% One line fields use \readifnotknown, which will only prompt
%% if the field has not already been set in the configuration file.
%%
\ifinteractive
  \typeout{^^JYour name:}
  \readifnotknown{name}
\else
  \ifx\name\undefined
    \def\name{ < ENTER YOUR NAME > }
  \fi
\fi


\ifinteractive
  \typeout{^^JYour Address (preferably email):}
  \readifnotknown{address}
\else
  \ifx\address\undefined
    \def\address{ < ENTER YOUR EMAIL ADDRESS > }
  \fi
\fi

\wmsg*{>Originator: \address \space(\name)}

%%
%% >Organisation: is really a GNATS multiline field
%% but we treat it as a one-line field.
%%
\wmsg*{>Organization: \ifx\organisation\undefined
                        \ifx\organization\undefined\else
                           \organization
                        \fi
                       \else
                         \organisation
                       \fi}


%%
%% Test which format is being used. These fields are completed
%% automatically even if the blank template is being produced.
%%

\wmsg*{>Environment:}
\wmsg*{ Hyphenation: \hyphenation}
\wmsg*{ \string\@TeXversion: \meaning\@TeXversion
        \ifx\@TeXversion\@@undefined
         \space (Standard setting for TeX3.141 and later)\fi}
\wmsg*{ \string\@currdir: \meaning\@currdir}
\wmsg*{ \string\input@path: \meaning\input@path
        \ifx\input@path\@@undefined
         \space (Standard setting)\fi}

\wmsg*{>Unformatted:}
\wmsg*{ *** Initex configuration files}
%%
%% These are in a comma separated list, so locally reset
%% \newlinechar
{\newlinechar`\,
 \wmsg*{\@inputfiles}}
\wmsg*{ ***}

\wmsg*{>Description:}
\typeout{}
\wmsg{Description of bug:}
\ifinteractive
  \typeout{%
    \@spaces The answer to this  question may take several lines.^^J%
    \@spaces (Each such line will be prompted by =>.)^^J%
    \@spaces Typing TWO consecutive blank lines terminates the answer.}
\else
\wmsg{ < ENTER BUG REPORT HERE >}
\fi
\wread



%%
%% insertion of the test file
%%



\ifinteractive
   \typeout{^^J%
 Name of a SHORT, SELF-CONTAINED file which indicates the problem:^^J%
 This file should be as small as possible (preferably < 60 lines)^^J%
 Any non-standard files that the test file uses should be included^^J%
 using the filecontents environment.^^J^^J%
%
 LaTeX will try to input this file, so give the full path^^J%
 if the file is not in the current directory.^^J^^J%
%
 If you are not reporting a bug, and there is therefore^^J%
 no test file, just hit <return>}
   \message{filename> }\read\m@ne to \filename
\else
   \def\filename{}
\fi

%% 
%% Try to find the .tex file and .log file
%%


\ifx\filename\@empty
  \ifx\LaTeX\undefinedcommand
  \else
    \ifinteractive
      \typeout{^^J^^JNo test file.^^J^^J%
      Three classes of report are supported:^^J^^J%
      0) sw-bug:^^J\@spaces
         Bug in the software, the report should include a test file.^^J
      1) doc-bug:^^J\@spaces
         Inaccuracies in the documentation.^^J
      2) change-request:^^J\@spaces 
         Not a bug, but rather a request for LaTeX to be changed.^^J}
      \message{Please select a category 0--2:  }
      \read\m@ne to \answer
      \ifx\answer\@empty
        \def\answer{-1}
      \fi
      \count@=\answer\relax
      \else
        \count@=\z@
      \fi
    \ifcase\count@
      \ifinteractive\wmsg{>Class: sw-bug}\fi
      \typeout{^^J! Please edit the message to add a test file and log!}
      \pause
      \wmsg*{^^J>How-To-Repeat:}
      \wmsg*{%
      Sample file which indicates the problem:^^J%
      ========================================^^J%
      \space< TEST FILE HERE >^^J%
      ^^J%
      The log file from running LaTeX on the sample:^^J%
      ==============================================^^J%
      \space< LOG FROM TEST FILE HERE >}
    \or
      \wmsg{>Class: doc-bug}
    \or
      \wmsg{>Class: change-request}
    \else
      \errhelp{Quit with `x' and then re-start latexbug}
      \def\badcategory{Only classes 0,1,2 are supported at this time}
      \errmessage{\badcategory}
    \fi
  \fi
\else

\filename@parse\filename

\IfFileExists{\filename}{\edef\samplefile{\filename}}{}

\IfFileExists{\filename@area\filename@base.log}
  {\edef\logfile{\filename@area\filename@base.log}}
  {\IfFileExists{\filename@area\filename@base.lis}
    {\edef\logfile{\filename@area\filename@base.lis}}
    {}}


%% 
%% The example file goes here:
%%
\wmsg*{^^J>How-To-Repeat:}

\wmsg*{^^J%
Sample file which indicates the problem:^^J%
========================================}

\ifx\samplefile\undefinedcommand
   \typeout{^^J%
      Sample file \filename\space not found.^^J%
      Please edit \jobname.msg to include the sample file.}
   \wmsg*{ < TEST FILE HERE >}
   \pause
\else
   \copytomsg{\samplefile}
   \ifnum\linecount>60
    \typeout{%
^^J%
!!! Your test file is \the\linecount\space lines long.^^J%
!!! Such a large test file causes us problems:^^J%
!!! * It makes it difficult to track down the error^^J%
!!! * It makes our database for storing reports unnecessarily large.^^J%
!!! ^^J%
!!! Please, if at all possible, cut down your test file to the^^J%
!!! smallest file that shows the behaviour.^^J}
   \pause
   \fi
\fi


%%
%% The log file goes here:
%%
\wmsg*{^^J%
The log file from running LaTeX on the sample:^^J%
==============================================}

\ifx\logfile\undefinedcommand
   \typeout{^^J%
      Log file \filename@area\filename@base.log not found.^^J%
      Please edit \jobname.msg to include the log file.}
   \wmsg*{ < < LOG FROM TEST FILE HERE >}
   \pause
\else
   \copytomsg{\logfile}
\fi

\fi


%%
%% Closing Banner.
%%
\typeout{^^J%
============================================================}

\ifinteractive
 \typeout{^^J%
   You may wish to make further changes to the bug report file:^^J%
   `\jobname.msg'^^J%
   using your editor.}
\else
 \typeout{^^J%
   A template for submitting bug reports has been left in the file:^^J%
   \jobname.msg^^J%
   Please use your editor to complete the file before submitting^^J%
   your report.}
\fi

\let\ifinteractivetrue\iftrue
\typeout{^^J%
  If you have access to email, please send `\jobname.msg' to:^^J%
  latex-bugs@latex-project.org  Please use the subject line:^^J%
  \@spaces Subject: \synopsis^^J%
^^J%
 (This subject will be used in all subsequent correspondence.)^^J%
^^J%
 To follow up an existing report, include the bug reference, e.g.^^J%
 "latex/1234: ", preceding the subject text.^^J%
^^J%
  Your message will be entered into a publicly readable database^^J%
  Accessable via the www (see bugs.txt for details).^^J%
  If do not wish this message made public, Edit the^^J%
  >Confidential: no^^J%
  field to  yes  before submitting this message.^^J%
^^J%
  Thank you for taking the time to submit a bug report.}

\wmsg*{^^J%
============================================================^^J
^^J%
   End of LaTeX2e bug report.^^J%
============================================================}

%%
%% Close the .msg output stream.
%%
\immediate\closeout\msg

%%
%% This is the TeX primitive \end command.
%%
\@@end
