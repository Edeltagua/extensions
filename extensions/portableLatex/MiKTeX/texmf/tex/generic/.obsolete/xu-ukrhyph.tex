% xu-ukrhyph.tex
% Wrapper for XeTeX to read Ukrainian hyphenation patterns
% Jonathan Kew, 2006-08-20
% Public domain

\begingroup

% Please uncomment the pattern value you need before
% creating a new format file containing Ukrainian hyphenation 
% patterns. 
% Note: `sm' offers most break points, so it is better 
% for narrow columns, `mp' offers least break points,
% and `st' and `mt' are in between. 

\ifx\Pattern\undefined
%\def\Pattern{sm}  %% by Andrij Shvaika, modern rules
%\def\Pattern{st}  %% by Andrij Shvaika, modern rules, 
                   %%   ``with removed suspicious breaks''
%\def\Pattern{mt}  %% by Maksym Polyakov old rules
\def\Pattern{mp}   %% by Maksym Polyakov old rules, breaking 
                   %%   into syllables according to phonetical principles. 
%\def\Pattern{fa}  %% derived from Russian patterns created by Dimitri Vulis
\fi

% For non-XeTeX use, also check the encoding options in ukrhyph.tex

\expandafter\ifx\csname XeTeXrevision\endcsname\relax

  \input ukrhyph.tex

\else

  \let\PATTERNS=\patterns
  \def\patterns{%
    \XeTeXinputencoding "bytes"
    \input xu-cp866nav.tex
    \lccode`\'=`\'
    \PATTERNS
  }

  \input ukrhyp\Pattern

\fi

\endgroup

\lefthyphenmin2 % settings copied from ukrhyph.tex
\righthyphenmin2

\endinput
