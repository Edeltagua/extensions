%%%%%%%%%%%%%%%%%%%%%%%%%%%%%%%%%%%%%%%%%%%%%%%%%%%%%%%%%%%%%%%%%%%%%%%
% animated metronome
% this code is based to 99.9 percent on the work by Manuel Luque
% (pstricks.blogspot.com)
%%%%%%%%%%%%%%%%%%%%%%%%%%%%%%%%%%%%%%%%%%%%%%%%%%%%%%%%%%%%%%%%%%%%%%%
\makeatletter
\pst@addfams{pst-metronome}
\define@key[psset]{pst-metronome}{theta0}{\def\psk@oscmetronomethetai{#1 }}
\psset[pst-metronome]{theta0=45} % position initiale du metronome
\define@key[psset]{pst-metronome}{M}{\def\psk@oscmetronometM{#1 }}
\psset[pst-metronome]{M=25} % masse du disque en g
\define@key[psset]{pst-metronome}{m}{\def\psk@oscmetronometm{#1 }}
\psset[pst-metronome]{m=6} % masse du curseur en g
\define@key[psset]{pst-metronome}{r}{\def\psk@oscmetronomer{#1 }}
\psset[pst-metronome]{r=1} % rayon du disque en cm
\define@key[psset]{pst-metronome}{x}{\def\psk@oscmetronomex{#1 }}
\psset[pst-metronome]{x=8.4} % position du curseur en cm par rapport � l'axe
\define@key[psset]{pst-metronome}{d}{\def\psk@oscmetronomed{#1 }}
\psset[pst-metronome]{d=3.2} % distance de l'axe au centre du disque en cm
\define@key[psset]{pst-metronome}{dt}{\def\psk@oscmetronomedt{#1 }}
\psset[pst-metronome]{dt=0.01} % pas pour RK4
\define@key[psset]{pst-metronome}{nT}{\def\psk@oscmetronomenT{#1 }}
\psset[pst-metronome]{nT=1} % nombre de p�riodes repr�sent�es
%---- calculer theta(t) et thetapoint(t) --------
\def\psmetronome{\pst@object{psmetronome}}
\def\psmetronome@i{%
\begingroup%
\use@par%
  \begin@SpecialObj%
  \pstVerb{%
 /deg2rad {180 div 3.14159 mul} def
 /rad2deg {180 mul 3.14159 div} def
 /gp 9.8 def % pesanteur
 /radius \psk@oscmetronomer 1e-2 mul def % en m
 /OA \psk@oscmetronomed 1e-2 mul def % distance de l'axe au centre du disque en m
 /xC \psk@oscmetronomex 1e-2 mul def % position du curseur en m par rapport � l'axe
 /theta0 \psk@oscmetronomethetai def % en degr�s
 /theta0rad theta0 deg2rad def % en radians
 /Md \psk@oscmetronometM 1e-3 mul def % en kg
 /mc \psk@oscmetronometm 1e-3 mul def % en kg
 /dt \psk@oscmetronomedt def
 /nT \psk@oscmetronomenT def
 % moment d'inertie du m�tronome
 % J=1/2M*R^2+M*a^2+m*x^2
 /Ji {0.5 Md mul radius dup mul mul Md OA dup mul mul add mc xC dup mul mul add} def
 /AT {4
      Ji
      gp Md OA mul mc xC mul sub mul
      div
      sqrt
      mul} def
 % Pour le calcul de la p�riode
 % coefficients de l'approximation polyn�miale du calcul
 % de l'int�grale elliptique
% coefficient pour le calcul de l'int�grale elliptique
        /m0 theta0 2 div sin def
        /m1 {1 m0 dup mul sub} def
        /m2 {m1 dup mul} def
        /m3 {m2 m1 mul} def
        /m4 {m2 dup mul} def
        /m_1 {1 m1 div} def
     /EllipticK {
        0.5
        0.12498593597 m1 mul add
        0.06880248576 m2 mul add
        0.03328355376 m3 mul add
        0.00441787012 m4 mul add
        m_1 ln mul
        1.38629436112 add
        0.09666344259 m1 mul add
        0.03590092383 m2 mul add
        0.03742563713 m3 mul add
        0.01451196212 m4 mul add
      } def
/Tm {AT EllipticK mul} def
% tableau des valeurs de theta(t)
    /W 0 def % vitesse angulaire
    /theta theta0 def
    /oscillateur {sin gp Md OA mul mc xC mul sub mul neg mul Ji div} def
    /j1 {W dt mul} def
    /k1 {theta oscillateur dt mul} def
    /j2 {W k1 2 div add dt mul} def
    /k2 {theta j1 2 div rad2deg add oscillateur dt mul} def
    /j3 {W k2 2 div add dt mul} def
    /k3 {theta j2 2 div rad2deg add oscillateur dt mul} def
    /j4 {W k3 add dt mul} def
    /k4 {theta j3 rad2deg add oscillateur dt mul} def
    /theta2 {theta j1 rad2deg 2 j2 rad2deg j3 rad2deg add mul add j4 rad2deg add 6 div add} def
/tabTheta [ % pour l'animation
    0 theta0 % date angle
dt dt Tm nT mul{ %
    theta2 %
    /W2 W k1 2 k2 k3 add mul add k4 add 6 div add def
    /theta theta2 def
    /W W2 def
    }  for
        ] def
/Nvaleurs tabTheta length 2 div cvi def
    /W 0 def % vitesse angulaire
    /theta theta0 def
/tabThetaGraph [ % pour le graphique theta(t)
    0 theta0  % date angle
0 dt Tm nT mul { % pop
    theta2 % 180 div 3.14159 mul
    /W2 W k1 2 k2 k3 add mul add k4 add 6 div add def
    /theta theta2 def
    /W W2 def
    }  for
        ] def
    /W 0 def % vitesse angulaire
    /theta theta0 def
/tabThetaPoint [ % pour le graphique thetapoint(t)
    0 0  % date angle
dt dt Tm nT mul { % pop
%    theta2 % 180 div 3.14159 mul
    /W2 W k1 2 k2 k3 add mul add k4 add 6 div add def
    W2
    /theta theta2 def
    /W W2 def
    }  for
        ] def
/tabXOSC [ % oscillations par min en fonction de x
0.5 0.1 12 {/xc exch def
 /xC xc 1e-2 mul def
 xc 60 Tm div % cvi
 } for
        ] def
/tabXbattements [ % battements par min en fonction de x
3 0.1 12 {/xc exch def
 /xC xc 1e-2 mul def
 xc 60 Tm div 2 mul % cvi
 } for
        ] def
% graduation T --> x
/tabXT [ % [T,x]
 40 1 220 {/batt exch def % battements
 /Tmetronome2 120 batt div dup mul def
 /A1 16 mc mul EllipticK dup mul mul def
 /B1 gp Tmetronome2 mul mc mul def
 /C1 gp Md mul OA mul Tmetronome2 mul neg
     8 Md mul radius dup mul mul 16 Md mul OA dup mul mul add EllipticK dup mul mul add def
 /Delta B1 dup mul 4 A1 mul C1 mul sub sqrt def
 /xC1 B1 neg Delta sub 2 div A1 div def
 /xC2 B1 neg Delta add 2 div A1 div def
 xC2 0 ge {/posC xC2 def}{/posC xC1 def} ifelse
 batt posC 1e2 mul
 } for
        ] def
/xT { % pour une valeur particuli�re battement -> position du curseur
    /batt exch def
    /Tmetronome2 120 batt div dup mul def
    /A1 16 mc mul EllipticK dup mul mul def
    /B1 gp Tmetronome2 mul mc mul def
    /C1 gp Md mul OA mul Tmetronome2 mul neg
     8 Md mul radius dup mul mul 16 Md mul OA dup mul mul add EllipticK dup mul mul add def
    /Delta B1 dup mul 4 A1 mul C1 mul sub sqrt def
    /xC1 B1 neg Delta sub 2 div A1 div def
    /xC2 B1 neg Delta add 2 div A1 div def
     xC2 0 ge {/posC xC2 def}{/posC xC1 def} ifelse
     posC 1e2 mul
 } def
/xC \psk@oscmetronomex 1e-2 mul def % position du curseur en m par rapport � l'axe
/Tm {AT EllipticK mul} def
  }%
  \end@SpecialObj%
\endgroup}
%
\def\psmetronomeA{\pst@object{psmetronomeA}}
\def\psmetronomeA@i{%
\begingroup%
\use@par%
  \begin@SpecialObj%
  \pstVerb{%
 /radius \psk@oscmetronomer 1e-2 mul def % en m
 /OA \psk@oscmetronomed 1e-2 mul def % distance de l'axe au centre du disque en m
 /xC \psk@oscmetronomex 1e-2 mul def % position du curseur en m par rapport � l'axe
  }%
\psframe[fillstyle=solid](! -0.075 \psk@oscmetronomed neg)(0.075,13)
\pscircle[fillstyle=solid,fillcolor={[rgb]{0.75 0.75 0.75}}](! 0 \psk@oscmetronomed neg){!radius 1e2 mul}
\pscircle[fillstyle=solid,linewidth=0.05](0,0){0.15}
\pscircle*[linecolor=red](0,0){0.05}
% curseur
\pspolygon[fillstyle=solid](! -0.25 \psk@oscmetronomex 0.5 sub)(! -0.5 \psk@oscmetronomex 0.5 add)(!-0.075 \psk@oscmetronomex 0.5 add)(!-0.075 \psk@oscmetronomex 0.5 sub)
\pspolygon[fillstyle=solid](! 0.25 \psk@oscmetronomex 0.5 sub)(! 0.5 \psk@oscmetronomex 0.5 add)(!0.075 \psk@oscmetronomex 0.5 add)(!0.075 \psk@oscmetronomex 0.5 sub)
\pspolygon[fillstyle=solid,fillcolor=gray](! -0.25 \psk@oscmetronomex 0.5 sub)(! -0.3 \psk@oscmetronomex 0.3 sub)(! -0.075 \psk@oscmetronomex 0.3 sub)(!-0.075 \psk@oscmetronomex 0.3 add)(!0.075 \psk@oscmetronomex 0.3 add)(!0.075 \psk@oscmetronomex 0.3 sub)(!0.3 \psk@oscmetronomex 0.3 sub)(!0.25 \psk@oscmetronomex 0.5 sub)
\pscircle[fillstyle=solid](!-0.125 \psk@oscmetronomex 0.4 sub){0.08}
\pscircle[fillstyle=solid](!0.125 \psk@oscmetronomex 0.4 sub){0.08}
% fin curseur
{\psset{linecolor=red}
\psline(!-.1 \psk@oscmetronomex)(!0.1 \psk@oscmetronomex)\psline(!0 \psk@oscmetronomex 0.1 sub)(!0 \psk@oscmetronomex 0.1 add)
\psline(! -.1 \psk@oscmetronomed neg)(!0.1 \psk@oscmetronomed neg)\psline(! 0 \psk@oscmetronomed neg 0.1 sub)(!0 \psk@oscmetronomed neg 0.1 add)}
\pnode(!0 \psk@oscmetronomex){C}% curseur
\pnode(! 0 \psk@oscmetronomed neg){D}% disque
\pstextpath[c](0,-2ex){\psarcn[linestyle=none](D){1}{180}{0}}{\small\textsf{\textbf{m e t r o n o m e}}}
\pstextpath[c](0,1ex){\psarc[linestyle=none](D){1}{180}{0}}{\small\textsf{\textbf{P S t r i c k s}}}
  \end@SpecialObj%
\endgroup}
\psmetronome%
\pstVerb{/tabTempos [40 42 44 46 48 50 52 54 46 58 60 63 66 69 72 76 80 84 88 92 96 100 104 108 112 116 120 126 132 138 144 152 160 168 176 184 192 200 208] def}%

\def\metronomebody{%
  \pspolygon[fillstyle=solid,linewidth=2\pslinewidth,linearc=0.5,fillcolor=yellow!30](-5,-4.5)(5,-4.5)(1,14)(-1,14)
  \psline(1.2,4.5)(1.2,12.5)
  \psline(-1.2,4.5)(-1.2,12.5)
  \multido{\i=0+2}{20}{%
     \pstVerb{/BATT tabTempos \i\space get def}
     \psline[linecolor=red](!1.2 BATT xT)(!0.7 BATT xT)
     \uput[r](!0.7 BATT xT){\psPrintValue[PSfont=Helvetica,fontscale=6]{BATT}}
  }%
  \multido{\i=1+2}{19}{%
     \pstVerb{/BATT tabTempos \i\space get def}
     \psline[linecolor=red](!-1.2 BATT xT)(!-0.7 BATT xT)
     \uput[r](!-1.3 BATT xT){\psPrintValue[PSfont=Helvetica,fontscale=6]{BATT}}}%
  \rput(!0 40 xT){\textsf{\tiny GRAVE}}%
  \rput(!0 46 xT){\textsf{\tiny LARGO}}%
  \rput(!0 52 xT){\textsf{\tiny LENTO}}%
  \rput(!0 58 xT){\textsf{\tiny ADAGIO}}%
  \rput(!0 60 xT){\textsf{\tiny LARGETTO}}%
  \rput(!0 66 xT){\textsf{\tiny ANDANTE}}%
  \rput(!0 76 xT){\textsf{\tiny ANDANTINO}}%
  \rput(!0 84 xT){\textsf{\tiny MODERATO}}%
  \rput(!0 108 xT){\textsf{\tiny ALLEGRETTO}}%
  \rput(!0 132 xT){\textsf{\tiny ALLEGRO}}%
  \rput(!0 160 xT){\textsf{\tiny VIVACE}}%
  \rput(!0 184 xT){\textsf{\tiny PRESTO}}%
  \rput(!0 200 xT){\textsf{\tiny PRESTISSIMO}}%
}

\def\pendulum#1{%
  \pstVerb{/iA #1\space def /date tabTheta iA get def /Theta tabTheta iA 1 add get def}%
  \rput{!Theta}{\psmetronomeA}%
}
\makeatother
